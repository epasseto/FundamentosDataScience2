
% Default to the notebook output style

    


% Inherit from the specified cell style.




    
\documentclass[11pt]{article}

    
    
    \usepackage[T1]{fontenc}
    % Nicer default font (+ math font) than Computer Modern for most use cases
    \usepackage{mathpazo}

    % Basic figure setup, for now with no caption control since it's done
    % automatically by Pandoc (which extracts ![](path) syntax from Markdown).
    \usepackage{graphicx}
    % We will generate all images so they have a width \maxwidth. This means
    % that they will get their normal width if they fit onto the page, but
    % are scaled down if they would overflow the margins.
    \makeatletter
    \def\maxwidth{\ifdim\Gin@nat@width>\linewidth\linewidth
    \else\Gin@nat@width\fi}
    \makeatother
    \let\Oldincludegraphics\includegraphics
    % Set max figure width to be 80% of text width, for now hardcoded.
    \renewcommand{\includegraphics}[1]{\Oldincludegraphics[width=.8\maxwidth]{#1}}
    % Ensure that by default, figures have no caption (until we provide a
    % proper Figure object with a Caption API and a way to capture that
    % in the conversion process - todo).
    \usepackage{caption}
    \DeclareCaptionLabelFormat{nolabel}{}
    \captionsetup{labelformat=nolabel}

    \usepackage{adjustbox} % Used to constrain images to a maximum size 
    \usepackage{xcolor} % Allow colors to be defined
    \usepackage{enumerate} % Needed for markdown enumerations to work
    \usepackage{geometry} % Used to adjust the document margins
    \usepackage{amsmath} % Equations
    \usepackage{amssymb} % Equations
    \usepackage{textcomp} % defines textquotesingle
    % Hack from http://tex.stackexchange.com/a/47451/13684:
    \AtBeginDocument{%
        \def\PYZsq{\textquotesingle}% Upright quotes in Pygmentized code
    }
    \usepackage{upquote} % Upright quotes for verbatim code
    \usepackage{eurosym} % defines \euro
    \usepackage[mathletters]{ucs} % Extended unicode (utf-8) support
    \usepackage[utf8x]{inputenc} % Allow utf-8 characters in the tex document
    \usepackage{fancyvrb} % verbatim replacement that allows latex
    \usepackage{grffile} % extends the file name processing of package graphics 
                         % to support a larger range 
    % The hyperref package gives us a pdf with properly built
    % internal navigation ('pdf bookmarks' for the table of contents,
    % internal cross-reference links, web links for URLs, etc.)
    \usepackage{hyperref}
    \usepackage{longtable} % longtable support required by pandoc >1.10
    \usepackage{booktabs}  % table support for pandoc > 1.12.2
    \usepackage[inline]{enumitem} % IRkernel/repr support (it uses the enumerate* environment)
    \usepackage[normalem]{ulem} % ulem is needed to support strikethroughs (\sout)
                                % normalem makes italics be italics, not underlines
    

    
    
    % Colors for the hyperref package
    \definecolor{urlcolor}{rgb}{0,.145,.698}
    \definecolor{linkcolor}{rgb}{.71,0.21,0.01}
    \definecolor{citecolor}{rgb}{.12,.54,.11}

    % ANSI colors
    \definecolor{ansi-black}{HTML}{3E424D}
    \definecolor{ansi-black-intense}{HTML}{282C36}
    \definecolor{ansi-red}{HTML}{E75C58}
    \definecolor{ansi-red-intense}{HTML}{B22B31}
    \definecolor{ansi-green}{HTML}{00A250}
    \definecolor{ansi-green-intense}{HTML}{007427}
    \definecolor{ansi-yellow}{HTML}{DDB62B}
    \definecolor{ansi-yellow-intense}{HTML}{B27D12}
    \definecolor{ansi-blue}{HTML}{208FFB}
    \definecolor{ansi-blue-intense}{HTML}{0065CA}
    \definecolor{ansi-magenta}{HTML}{D160C4}
    \definecolor{ansi-magenta-intense}{HTML}{A03196}
    \definecolor{ansi-cyan}{HTML}{60C6C8}
    \definecolor{ansi-cyan-intense}{HTML}{258F8F}
    \definecolor{ansi-white}{HTML}{C5C1B4}
    \definecolor{ansi-white-intense}{HTML}{A1A6B2}

    % commands and environments needed by pandoc snippets
    % extracted from the output of `pandoc -s`
    \providecommand{\tightlist}{%
      \setlength{\itemsep}{0pt}\setlength{\parskip}{0pt}}
    \DefineVerbatimEnvironment{Highlighting}{Verbatim}{commandchars=\\\{\}}
    % Add ',fontsize=\small' for more characters per line
    \newenvironment{Shaded}{}{}
    \newcommand{\KeywordTok}[1]{\textcolor[rgb]{0.00,0.44,0.13}{\textbf{{#1}}}}
    \newcommand{\DataTypeTok}[1]{\textcolor[rgb]{0.56,0.13,0.00}{{#1}}}
    \newcommand{\DecValTok}[1]{\textcolor[rgb]{0.25,0.63,0.44}{{#1}}}
    \newcommand{\BaseNTok}[1]{\textcolor[rgb]{0.25,0.63,0.44}{{#1}}}
    \newcommand{\FloatTok}[1]{\textcolor[rgb]{0.25,0.63,0.44}{{#1}}}
    \newcommand{\CharTok}[1]{\textcolor[rgb]{0.25,0.44,0.63}{{#1}}}
    \newcommand{\StringTok}[1]{\textcolor[rgb]{0.25,0.44,0.63}{{#1}}}
    \newcommand{\CommentTok}[1]{\textcolor[rgb]{0.38,0.63,0.69}{\textit{{#1}}}}
    \newcommand{\OtherTok}[1]{\textcolor[rgb]{0.00,0.44,0.13}{{#1}}}
    \newcommand{\AlertTok}[1]{\textcolor[rgb]{1.00,0.00,0.00}{\textbf{{#1}}}}
    \newcommand{\FunctionTok}[1]{\textcolor[rgb]{0.02,0.16,0.49}{{#1}}}
    \newcommand{\RegionMarkerTok}[1]{{#1}}
    \newcommand{\ErrorTok}[1]{\textcolor[rgb]{1.00,0.00,0.00}{\textbf{{#1}}}}
    \newcommand{\NormalTok}[1]{{#1}}
    
    % Additional commands for more recent versions of Pandoc
    \newcommand{\ConstantTok}[1]{\textcolor[rgb]{0.53,0.00,0.00}{{#1}}}
    \newcommand{\SpecialCharTok}[1]{\textcolor[rgb]{0.25,0.44,0.63}{{#1}}}
    \newcommand{\VerbatimStringTok}[1]{\textcolor[rgb]{0.25,0.44,0.63}{{#1}}}
    \newcommand{\SpecialStringTok}[1]{\textcolor[rgb]{0.73,0.40,0.53}{{#1}}}
    \newcommand{\ImportTok}[1]{{#1}}
    \newcommand{\DocumentationTok}[1]{\textcolor[rgb]{0.73,0.13,0.13}{\textit{{#1}}}}
    \newcommand{\AnnotationTok}[1]{\textcolor[rgb]{0.38,0.63,0.69}{\textbf{\textit{{#1}}}}}
    \newcommand{\CommentVarTok}[1]{\textcolor[rgb]{0.38,0.63,0.69}{\textbf{\textit{{#1}}}}}
    \newcommand{\VariableTok}[1]{\textcolor[rgb]{0.10,0.09,0.49}{{#1}}}
    \newcommand{\ControlFlowTok}[1]{\textcolor[rgb]{0.00,0.44,0.13}{\textbf{{#1}}}}
    \newcommand{\OperatorTok}[1]{\textcolor[rgb]{0.40,0.40,0.40}{{#1}}}
    \newcommand{\BuiltInTok}[1]{{#1}}
    \newcommand{\ExtensionTok}[1]{{#1}}
    \newcommand{\PreprocessorTok}[1]{\textcolor[rgb]{0.74,0.48,0.00}{{#1}}}
    \newcommand{\AttributeTok}[1]{\textcolor[rgb]{0.49,0.56,0.16}{{#1}}}
    \newcommand{\InformationTok}[1]{\textcolor[rgb]{0.38,0.63,0.69}{\textbf{\textit{{#1}}}}}
    \newcommand{\WarningTok}[1]{\textcolor[rgb]{0.38,0.63,0.69}{\textbf{\textit{{#1}}}}}
    
    
    % Define a nice break command that doesn't care if a line doesn't already
    % exist.
    \def\br{\hspace*{\fill} \\* }
    % Math Jax compatability definitions
    \def\gt{>}
    \def\lt{<}
    % Document parameters
    \title{AnalyzeABTestResults3}
    
    
    

    % Pygments definitions
    
\makeatletter
\def\PY@reset{\let\PY@it=\relax \let\PY@bf=\relax%
    \let\PY@ul=\relax \let\PY@tc=\relax%
    \let\PY@bc=\relax \let\PY@ff=\relax}
\def\PY@tok#1{\csname PY@tok@#1\endcsname}
\def\PY@toks#1+{\ifx\relax#1\empty\else%
    \PY@tok{#1}\expandafter\PY@toks\fi}
\def\PY@do#1{\PY@bc{\PY@tc{\PY@ul{%
    \PY@it{\PY@bf{\PY@ff{#1}}}}}}}
\def\PY#1#2{\PY@reset\PY@toks#1+\relax+\PY@do{#2}}

\expandafter\def\csname PY@tok@w\endcsname{\def\PY@tc##1{\textcolor[rgb]{0.73,0.73,0.73}{##1}}}
\expandafter\def\csname PY@tok@c\endcsname{\let\PY@it=\textit\def\PY@tc##1{\textcolor[rgb]{0.25,0.50,0.50}{##1}}}
\expandafter\def\csname PY@tok@cp\endcsname{\def\PY@tc##1{\textcolor[rgb]{0.74,0.48,0.00}{##1}}}
\expandafter\def\csname PY@tok@k\endcsname{\let\PY@bf=\textbf\def\PY@tc##1{\textcolor[rgb]{0.00,0.50,0.00}{##1}}}
\expandafter\def\csname PY@tok@kp\endcsname{\def\PY@tc##1{\textcolor[rgb]{0.00,0.50,0.00}{##1}}}
\expandafter\def\csname PY@tok@kt\endcsname{\def\PY@tc##1{\textcolor[rgb]{0.69,0.00,0.25}{##1}}}
\expandafter\def\csname PY@tok@o\endcsname{\def\PY@tc##1{\textcolor[rgb]{0.40,0.40,0.40}{##1}}}
\expandafter\def\csname PY@tok@ow\endcsname{\let\PY@bf=\textbf\def\PY@tc##1{\textcolor[rgb]{0.67,0.13,1.00}{##1}}}
\expandafter\def\csname PY@tok@nb\endcsname{\def\PY@tc##1{\textcolor[rgb]{0.00,0.50,0.00}{##1}}}
\expandafter\def\csname PY@tok@nf\endcsname{\def\PY@tc##1{\textcolor[rgb]{0.00,0.00,1.00}{##1}}}
\expandafter\def\csname PY@tok@nc\endcsname{\let\PY@bf=\textbf\def\PY@tc##1{\textcolor[rgb]{0.00,0.00,1.00}{##1}}}
\expandafter\def\csname PY@tok@nn\endcsname{\let\PY@bf=\textbf\def\PY@tc##1{\textcolor[rgb]{0.00,0.00,1.00}{##1}}}
\expandafter\def\csname PY@tok@ne\endcsname{\let\PY@bf=\textbf\def\PY@tc##1{\textcolor[rgb]{0.82,0.25,0.23}{##1}}}
\expandafter\def\csname PY@tok@nv\endcsname{\def\PY@tc##1{\textcolor[rgb]{0.10,0.09,0.49}{##1}}}
\expandafter\def\csname PY@tok@no\endcsname{\def\PY@tc##1{\textcolor[rgb]{0.53,0.00,0.00}{##1}}}
\expandafter\def\csname PY@tok@nl\endcsname{\def\PY@tc##1{\textcolor[rgb]{0.63,0.63,0.00}{##1}}}
\expandafter\def\csname PY@tok@ni\endcsname{\let\PY@bf=\textbf\def\PY@tc##1{\textcolor[rgb]{0.60,0.60,0.60}{##1}}}
\expandafter\def\csname PY@tok@na\endcsname{\def\PY@tc##1{\textcolor[rgb]{0.49,0.56,0.16}{##1}}}
\expandafter\def\csname PY@tok@nt\endcsname{\let\PY@bf=\textbf\def\PY@tc##1{\textcolor[rgb]{0.00,0.50,0.00}{##1}}}
\expandafter\def\csname PY@tok@nd\endcsname{\def\PY@tc##1{\textcolor[rgb]{0.67,0.13,1.00}{##1}}}
\expandafter\def\csname PY@tok@s\endcsname{\def\PY@tc##1{\textcolor[rgb]{0.73,0.13,0.13}{##1}}}
\expandafter\def\csname PY@tok@sd\endcsname{\let\PY@it=\textit\def\PY@tc##1{\textcolor[rgb]{0.73,0.13,0.13}{##1}}}
\expandafter\def\csname PY@tok@si\endcsname{\let\PY@bf=\textbf\def\PY@tc##1{\textcolor[rgb]{0.73,0.40,0.53}{##1}}}
\expandafter\def\csname PY@tok@se\endcsname{\let\PY@bf=\textbf\def\PY@tc##1{\textcolor[rgb]{0.73,0.40,0.13}{##1}}}
\expandafter\def\csname PY@tok@sr\endcsname{\def\PY@tc##1{\textcolor[rgb]{0.73,0.40,0.53}{##1}}}
\expandafter\def\csname PY@tok@ss\endcsname{\def\PY@tc##1{\textcolor[rgb]{0.10,0.09,0.49}{##1}}}
\expandafter\def\csname PY@tok@sx\endcsname{\def\PY@tc##1{\textcolor[rgb]{0.00,0.50,0.00}{##1}}}
\expandafter\def\csname PY@tok@m\endcsname{\def\PY@tc##1{\textcolor[rgb]{0.40,0.40,0.40}{##1}}}
\expandafter\def\csname PY@tok@gh\endcsname{\let\PY@bf=\textbf\def\PY@tc##1{\textcolor[rgb]{0.00,0.00,0.50}{##1}}}
\expandafter\def\csname PY@tok@gu\endcsname{\let\PY@bf=\textbf\def\PY@tc##1{\textcolor[rgb]{0.50,0.00,0.50}{##1}}}
\expandafter\def\csname PY@tok@gd\endcsname{\def\PY@tc##1{\textcolor[rgb]{0.63,0.00,0.00}{##1}}}
\expandafter\def\csname PY@tok@gi\endcsname{\def\PY@tc##1{\textcolor[rgb]{0.00,0.63,0.00}{##1}}}
\expandafter\def\csname PY@tok@gr\endcsname{\def\PY@tc##1{\textcolor[rgb]{1.00,0.00,0.00}{##1}}}
\expandafter\def\csname PY@tok@ge\endcsname{\let\PY@it=\textit}
\expandafter\def\csname PY@tok@gs\endcsname{\let\PY@bf=\textbf}
\expandafter\def\csname PY@tok@gp\endcsname{\let\PY@bf=\textbf\def\PY@tc##1{\textcolor[rgb]{0.00,0.00,0.50}{##1}}}
\expandafter\def\csname PY@tok@go\endcsname{\def\PY@tc##1{\textcolor[rgb]{0.53,0.53,0.53}{##1}}}
\expandafter\def\csname PY@tok@gt\endcsname{\def\PY@tc##1{\textcolor[rgb]{0.00,0.27,0.87}{##1}}}
\expandafter\def\csname PY@tok@err\endcsname{\def\PY@bc##1{\setlength{\fboxsep}{0pt}\fcolorbox[rgb]{1.00,0.00,0.00}{1,1,1}{\strut ##1}}}
\expandafter\def\csname PY@tok@kc\endcsname{\let\PY@bf=\textbf\def\PY@tc##1{\textcolor[rgb]{0.00,0.50,0.00}{##1}}}
\expandafter\def\csname PY@tok@kd\endcsname{\let\PY@bf=\textbf\def\PY@tc##1{\textcolor[rgb]{0.00,0.50,0.00}{##1}}}
\expandafter\def\csname PY@tok@kn\endcsname{\let\PY@bf=\textbf\def\PY@tc##1{\textcolor[rgb]{0.00,0.50,0.00}{##1}}}
\expandafter\def\csname PY@tok@kr\endcsname{\let\PY@bf=\textbf\def\PY@tc##1{\textcolor[rgb]{0.00,0.50,0.00}{##1}}}
\expandafter\def\csname PY@tok@bp\endcsname{\def\PY@tc##1{\textcolor[rgb]{0.00,0.50,0.00}{##1}}}
\expandafter\def\csname PY@tok@fm\endcsname{\def\PY@tc##1{\textcolor[rgb]{0.00,0.00,1.00}{##1}}}
\expandafter\def\csname PY@tok@vc\endcsname{\def\PY@tc##1{\textcolor[rgb]{0.10,0.09,0.49}{##1}}}
\expandafter\def\csname PY@tok@vg\endcsname{\def\PY@tc##1{\textcolor[rgb]{0.10,0.09,0.49}{##1}}}
\expandafter\def\csname PY@tok@vi\endcsname{\def\PY@tc##1{\textcolor[rgb]{0.10,0.09,0.49}{##1}}}
\expandafter\def\csname PY@tok@vm\endcsname{\def\PY@tc##1{\textcolor[rgb]{0.10,0.09,0.49}{##1}}}
\expandafter\def\csname PY@tok@sa\endcsname{\def\PY@tc##1{\textcolor[rgb]{0.73,0.13,0.13}{##1}}}
\expandafter\def\csname PY@tok@sb\endcsname{\def\PY@tc##1{\textcolor[rgb]{0.73,0.13,0.13}{##1}}}
\expandafter\def\csname PY@tok@sc\endcsname{\def\PY@tc##1{\textcolor[rgb]{0.73,0.13,0.13}{##1}}}
\expandafter\def\csname PY@tok@dl\endcsname{\def\PY@tc##1{\textcolor[rgb]{0.73,0.13,0.13}{##1}}}
\expandafter\def\csname PY@tok@s2\endcsname{\def\PY@tc##1{\textcolor[rgb]{0.73,0.13,0.13}{##1}}}
\expandafter\def\csname PY@tok@sh\endcsname{\def\PY@tc##1{\textcolor[rgb]{0.73,0.13,0.13}{##1}}}
\expandafter\def\csname PY@tok@s1\endcsname{\def\PY@tc##1{\textcolor[rgb]{0.73,0.13,0.13}{##1}}}
\expandafter\def\csname PY@tok@mb\endcsname{\def\PY@tc##1{\textcolor[rgb]{0.40,0.40,0.40}{##1}}}
\expandafter\def\csname PY@tok@mf\endcsname{\def\PY@tc##1{\textcolor[rgb]{0.40,0.40,0.40}{##1}}}
\expandafter\def\csname PY@tok@mh\endcsname{\def\PY@tc##1{\textcolor[rgb]{0.40,0.40,0.40}{##1}}}
\expandafter\def\csname PY@tok@mi\endcsname{\def\PY@tc##1{\textcolor[rgb]{0.40,0.40,0.40}{##1}}}
\expandafter\def\csname PY@tok@il\endcsname{\def\PY@tc##1{\textcolor[rgb]{0.40,0.40,0.40}{##1}}}
\expandafter\def\csname PY@tok@mo\endcsname{\def\PY@tc##1{\textcolor[rgb]{0.40,0.40,0.40}{##1}}}
\expandafter\def\csname PY@tok@ch\endcsname{\let\PY@it=\textit\def\PY@tc##1{\textcolor[rgb]{0.25,0.50,0.50}{##1}}}
\expandafter\def\csname PY@tok@cm\endcsname{\let\PY@it=\textit\def\PY@tc##1{\textcolor[rgb]{0.25,0.50,0.50}{##1}}}
\expandafter\def\csname PY@tok@cpf\endcsname{\let\PY@it=\textit\def\PY@tc##1{\textcolor[rgb]{0.25,0.50,0.50}{##1}}}
\expandafter\def\csname PY@tok@c1\endcsname{\let\PY@it=\textit\def\PY@tc##1{\textcolor[rgb]{0.25,0.50,0.50}{##1}}}
\expandafter\def\csname PY@tok@cs\endcsname{\let\PY@it=\textit\def\PY@tc##1{\textcolor[rgb]{0.25,0.50,0.50}{##1}}}

\def\PYZbs{\char`\\}
\def\PYZus{\char`\_}
\def\PYZob{\char`\{}
\def\PYZcb{\char`\}}
\def\PYZca{\char`\^}
\def\PYZam{\char`\&}
\def\PYZlt{\char`\<}
\def\PYZgt{\char`\>}
\def\PYZsh{\char`\#}
\def\PYZpc{\char`\%}
\def\PYZdl{\char`\$}
\def\PYZhy{\char`\-}
\def\PYZsq{\char`\'}
\def\PYZdq{\char`\"}
\def\PYZti{\char`\~}
% for compatibility with earlier versions
\def\PYZat{@}
\def\PYZlb{[}
\def\PYZrb{]}
\makeatother


    % Exact colors from NB
    \definecolor{incolor}{rgb}{0.0, 0.0, 0.5}
    \definecolor{outcolor}{rgb}{0.545, 0.0, 0.0}



    
    % Prevent overflowing lines due to hard-to-break entities
    \sloppy 
    % Setup hyperref package
    \hypersetup{
      breaklinks=true,  % so long urls are correctly broken across lines
      colorlinks=true,
      urlcolor=urlcolor,
      linkcolor=linkcolor,
      citecolor=citecolor,
      }
    % Slightly bigger margins than the latex defaults
    
    \geometry{verbose,tmargin=1in,bmargin=1in,lmargin=1in,rmargin=1in}
    
    

    \begin{document}
    
    
    \maketitle
    
    

    
    \hypertarget{analyze-ab-test-results}{%
\subsection{Analyze A/B Test Results}\label{analyze-ab-test-results}}

This project will assure you have mastered the subjects covered in the
statistics lessons. The hope is to have this project be as comprehensive
of these topics as possible. Good luck!

\hypertarget{table-of-contents}{%
\subsection{Table of Contents}\label{table-of-contents}}

\begin{itemize}
\tightlist
\item
  Section \ref{intro}
\item
  Section \ref{probability}
\item
  Section \ref{ab_test}
\item
  Section \ref{regression}
\end{itemize}

 \#\#\# Introduction

A/B tests are very commonly performed by data analysts and data
scientists. It is important that you get some practice working with the
difficulties of these

For this project, you will be working to understand the results of an
A/B test run by an e-commerce website. Your goal is to work through this
notebook to help the company understand if they should implement the new
page, keep the old page, or perhaps run the experiment longer to make
their decision.

\textbf{As you work through this notebook, follow along in the classroom
and answer the corresponding quiz questions associated with each
question.} The labels for each classroom concept are provided for each
question. This will assure you are on the right track as you work
through the project, and you can feel more confident in your final
submission meeting the criteria. As a final check, assure you meet all
the criteria on the
\href{https://review.udacity.com/\#!/projects/37e27304-ad47-4eb0-a1ab-8c12f60e43d0/rubric}{RUBRIC}.

 \#\#\#\# Part I - Probability

To get started, let's import our libraries.

    \begin{Verbatim}[commandchars=\\\{\}]
{\color{incolor}In [{\color{incolor}1}]:} \PY{k+kn}{import} \PY{n+nn}{pandas} \PY{k}{as} \PY{n+nn}{pd}
        \PY{k+kn}{import} \PY{n+nn}{numpy} \PY{k}{as} \PY{n+nn}{np}
        \PY{k+kn}{import} \PY{n+nn}{random}
        \PY{k+kn}{import} \PY{n+nn}{matplotlib}\PY{n+nn}{.}\PY{n+nn}{pyplot} \PY{k}{as} \PY{n+nn}{plt}
        \PY{o}{\PYZpc{}}\PY{k}{matplotlib} inline
        \PY{c+c1}{\PYZsh{}We are setting the seed to assure you get the same answers on quizzes as we set up}
        \PY{n}{random}\PY{o}{.}\PY{n}{seed}\PY{p}{(}\PY{l+m+mi}{42}\PY{p}{)}
\end{Verbatim}


    \texttt{1.} Now, read in the \texttt{ab\_data.csv} data. Store it in
\texttt{df}. \textbf{Use your dataframe to answer the questions in Quiz
1 of the classroom.}

\begin{enumerate}
\def\labelenumi{\alph{enumi}.}
\tightlist
\item
  Read in the dataset and take a look at the top few rows here:
\end{enumerate}

    \emph{Eu não vou mudar os nomes das colunas (apesar de ter ficado na
tentação), pois parece um exercício guiado}

    \begin{Verbatim}[commandchars=\\\{\}]
{\color{incolor}In [{\color{incolor}2}]:} \PY{n}{dfdados} \PY{o}{=} \PY{n}{pd}\PY{o}{.}\PY{n}{read\PYZus{}csv}\PY{p}{(}\PY{l+s+s1}{\PYZsq{}}\PY{l+s+s1}{ab\PYZus{}data.csv}\PY{l+s+s1}{\PYZsq{}}\PY{p}{,} \PY{n}{sep}\PY{o}{=}\PY{l+s+s1}{\PYZsq{}}\PY{l+s+s1}{,}\PY{l+s+s1}{\PYZsq{}}\PY{p}{,} \PY{n}{encoding}\PY{o}{=}\PY{l+s+s1}{\PYZsq{}}\PY{l+s+s1}{utf\PYZhy{}8}\PY{l+s+s1}{\PYZsq{}}\PY{p}{,} \PY{n}{index\PYZus{}col}\PY{o}{=}\PY{l+s+s1}{\PYZsq{}}\PY{l+s+s1}{user\PYZus{}id}\PY{l+s+s1}{\PYZsq{}}\PY{p}{)}
\end{Verbatim}


    \emph{Interpretando o que temos aqui. É um teste de duas páginas de
Webpage, como aquelas da empresa \textbf{Audacity}. Então usuários
encontram a empresa num mecanismo de busca e são direcionados a uma
página oferecendo cursos. O detalhe é que a busca tem um
\textbf{forking} e ela conduz alguns usuários para a página antiga
(\textbf{grupo de controle}) e outros para a página novo (\textbf{grupo
de tratamento}. O usuário lê algumas coisas e se se interessar, irá
clicar numa espécie de botão (``adquirir o produto'', o que é chamado de
\textbf{converted}).}

O dataset grava as seguintes informações:

\begin{itemize}
\item
  id do usuário (para saber se não é a mesma pessoa que está entrando
  duas vezes - pode ter havido um erro no carregamento da página, etc..)
  - eu quero trabalhar com IDs \textbf{únicos}
\item
  timestamp (um formato de data e hora do acesso)
\item
  grupo (\textbf{controle} ou \textbf{tratamento})
\item
  página acessada (\textbf{antiga} ou \textbf{nova}) - informação
  redundante, pois a do grupo deveria suprir isso!
\item
  convertido (\textbf{zero} ou \textbf{um}) - a pessoa se
  \textbf{converteu}, ou seja, comprou o produto?
\end{itemize}

Uma H0 típica de um caso desses seria:

\begin{itemize}
\tightlist
\item
  a taxa de conversão da página \textbf{antiga} é igual ou maior do que
  a taxa de conversão da página \textbf{nova}
\end{itemize}

Uma H1 típica:

\begin{itemize}
\tightlist
\item
  a taxa de conversão da página \textbf{nova} é maior do que a taxa de
  conversão da página \textbf{antiga}
\end{itemize}

    \begin{Verbatim}[commandchars=\\\{\}]
{\color{incolor}In [{\color{incolor}134}]:} \PY{n}{dfdados}\PY{o}{.}\PY{n}{head}\PY{p}{(}\PY{l+m+mi}{5}\PY{p}{)}
\end{Verbatim}


\begin{Verbatim}[commandchars=\\\{\}]
{\color{outcolor}Out[{\color{outcolor}134}]:}                           timestamp      group landing\_page  converted
          user\_id                                                               
          851104   2017-01-21 22:11:48.556739    control     old\_page          0
          804228   2017-01-12 08:01:45.159739    control     old\_page          0
          661590   2017-01-11 16:55:06.154213  treatment     new\_page          0
          853541   2017-01-08 18:28:03.143765  treatment     new\_page          0
          864975   2017-01-21 01:52:26.210827    control     old\_page          1
\end{Verbatim}
            
    \begin{enumerate}
\def\labelenumi{\alph{enumi}.}
\setcounter{enumi}{1}
\tightlist
\item
  Use the below cell to find the number of rows in the dataset.
\end{enumerate}

    \emph{Costumeiramente eu faço isso:}

    \begin{Verbatim}[commandchars=\\\{\}]
{\color{incolor}In [{\color{incolor}135}]:} \PY{n}{dfdados}\PY{o}{.}\PY{n}{info}\PY{p}{(}\PY{p}{)}
\end{Verbatim}


    \begin{Verbatim}[commandchars=\\\{\}]
<class 'pandas.core.frame.DataFrame'>
Int64Index: 294478 entries, 851104 to 715931
Data columns (total 4 columns):
timestamp       294478 non-null object
group           294478 non-null object
landing\_page    294478 non-null object
converted       294478 non-null int64
dtypes: int64(1), object(3)
memory usage: 11.2+ MB

    \end{Verbatim}

    \emph{Mais elegante\ldots{}}

    \begin{Verbatim}[commandchars=\\\{\}]
{\color{incolor}In [{\color{incolor}139}]:} \PY{n+nb}{print} \PY{p}{(}\PY{l+s+s1}{\PYZsq{}}\PY{l+s+s1}{São }\PY{l+s+si}{\PYZob{}\PYZcb{}}\PY{l+s+s1}{ linhas no dataset}\PY{l+s+s1}{\PYZsq{}}\PY{o}{.}\PY{n}{format}\PY{p}{(}\PY{n}{dfdados}\PY{o}{.}\PY{n}{shape}\PY{p}{[}\PY{l+m+mi}{0}\PY{p}{]}\PY{p}{)}\PY{p}{)}
\end{Verbatim}


    \begin{Verbatim}[commandchars=\\\{\}]
São 294478 linhas no dataset

    \end{Verbatim}

    \begin{enumerate}
\def\labelenumi{\alph{enumi}.}
\setcounter{enumi}{2}
\tightlist
\item
  The number of unique users in the dataset.
\end{enumerate}

\emph{Nada como Stack Overflow - eles recomendam fazer literalmente
algumas operações, como dropna=True (isso pode me fazer evitar erros no
futuro!)}

\emph{Aparentemente alguns usuários visitaram o site várias vezes}

    \begin{Verbatim}[commandchars=\\\{\}]
{\color{incolor}In [{\color{incolor}4}]:} \PY{n}{ususing} \PY{o}{=} \PY{n}{dfdados}\PY{o}{.}\PY{n}{index}\PY{o}{.}\PY{n}{nunique}\PY{p}{(}\PY{p}{)} \PY{c+c1}{\PYZsh{}dropna=True aqui realmente não existem NaNs}
        
        \PY{n+nb}{print} \PY{p}{(}\PY{l+s+s1}{\PYZsq{}}\PY{l+s+s1}{São }\PY{l+s+si}{\PYZob{}\PYZcb{}}\PY{l+s+s1}{ usuários singulares}\PY{l+s+s1}{\PYZsq{}}\PY{o}{.}\PY{n}{format}\PY{p}{(}\PY{n}{ususing}\PY{p}{)}\PY{p}{)}
\end{Verbatim}


    \begin{Verbatim}[commandchars=\\\{\}]
São 290584 usuários singulares

    \end{Verbatim}

    \hypertarget{section}{%
\subsubsection{2/5}\label{section}}

\begin{enumerate}
\def\labelenumi{\alph{enumi}.}
\setcounter{enumi}{3}
\tightlist
\item
  The proportion of users converted.
\end{enumerate}

    Não ficou \textbf{perfeito}:

\begin{itemize}
\tightlist
\item
  eu tenho alguns usuários repetidos, o cara entrou uma, duas vezes e
  acabou comprando ou não\ldots{} na terceira entrada!
\end{itemize}

\emph{Por que estou insistindo neste pequeno problema deste dataset? Por
uma razão simples, se seguirmos a \textbf{ortodoxia} da análise
Estatística, precisaríamos \textbf{eliminar} as diversas entradas de um
usuário singular! Por que? Por uma razão muito simples, uma das
premissas da inferência é de que os eventos devem ser
\textbf{independentes}!}

\emph{Depois dá para verificar que este pequeno desvio é passável, pois
foi realmente um pequeno número de usuários que fizeram mais de um
acesso (e possivelmente à página \textbf{nova} e \textbf{antiga}). Mas
funciona assim: se eu sou um usuário \textbf{singular} e muito bem,
entro numa máquina\ldots{} ``hmmm, vi, não animei (10\%
convencido)''\ldots{} entro de novo\ldots{} ``hmmm, sei, ainda não (30\%
convencido)''\ldots{} e entro a terceira vez e faço a compra (100\%
convencido)! Sabe o que aconteceu? Isso não é como jogar uma moeda
\textbf{justa}, pois neste caso, dado um usuário, a entrada anterior
\textbf{condiciona} a posterior! (pois ele possui \textbf{memória})}

\emph{Tudo bem, este é apenas um exercício\ldots{} mas se formos tratar
dados de verdade, temos que atentar aos pequenos detalhes\ldots{} e que
ao final podem ser \textbf{fatais}!}

\emph{Mais um comentário, como faríamos em \textbf{Física}? Pegamos
sempre o \textbf{último} resultado para um indivíduo\ldots{} Dizemos que
ele colapsou em \(n(t)\)\ldots{} mas isso é mera curiosidade\ldots{}}

    \begin{Verbatim}[commandchars=\\\{\}]
{\color{incolor}In [{\color{incolor}163}]:} \PY{n}{convertidos} \PY{o}{=} \PY{n}{dfdados}\PY{o}{.}\PY{n}{converted}
          \PY{c+c1}{\PYZsh{}convertidos}
          \PY{n+nb}{print}\PY{p}{(}\PY{l+s+s1}{\PYZsq{}}\PY{l+s+s1}{Aderiram em média }\PY{l+s+si}{\PYZob{}\PYZcb{}}\PY{l+s+s1}{ usuários}\PY{l+s+s1}{\PYZsq{}}\PY{o}{.}\PY{n}{format}\PY{p}{(}\PY{n}{convertidos}\PY{o}{.}\PY{n}{mean}\PY{p}{(}\PY{p}{)}\PY{p}{)}\PY{p}{)}
\end{Verbatim}


    \begin{Verbatim}[commandchars=\\\{\}]
Aderiram em média 0.11965919355605512 usuários

    \end{Verbatim}

    \emph{Nada como praticar jeitos diferentes\ldots{} de grafia\ldots{} (já
me embaralhei demais lendo códigos de terceiros e pensando que eram
coisas diferentes!)}

Eu gosto mais de escrever o código o mais \textbf{explícito} possível.
Me confundo menos:

\begin{itemize}
\item
  resumo da ópera, a diferença neste caso ocorreu na casa
  \textbf{centesimal} da porcentagem (o que seria desprezível)
\item
  no entando, o de baixo deve ser considerado o resultado \textbf{real}
  (o que usamos está ligeiramente distorcido pelas diversas entradas de
  um mesmo usuário)
\end{itemize}

    \begin{Verbatim}[commandchars=\\\{\}]
{\color{incolor}In [{\color{incolor}180}]:} \PY{n}{compras} \PY{o}{=} \PY{n}{dfdados}\PY{p}{[}\PY{l+s+s1}{\PYZsq{}}\PY{l+s+s1}{converted}\PY{l+s+s1}{\PYZsq{}}\PY{p}{]}\PY{o}{.}\PY{n}{groupby}\PY{p}{(}\PY{n}{dfdados}\PY{o}{.}\PY{n}{index}\PY{p}{)}\PY{o}{.}\PY{n}{first}\PY{p}{(}\PY{p}{)} \PY{c+c1}{\PYZsh{} index.duplicated()}
          \PY{c+c1}{\PYZsh{}df3 = df3[\PYZti{}df3.index.duplicated(keep=\PYZsq{}first\PYZsq{})]}
          \PY{n+nb}{len}\PY{p}{(}\PY{n}{compras}\PY{p}{)}
          \PY{n+nb}{print}\PY{p}{(}\PY{l+s+s1}{\PYZsq{}}\PY{l+s+s1}{Aderiram em média }\PY{l+s+si}{\PYZob{}\PYZcb{}}\PY{l+s+s1}{ usuários (resultado mais preciso)}\PY{l+s+s1}{\PYZsq{}}\PY{o}{.}\PY{n}{format}\PY{p}{(}\PY{n}{compras}\PY{o}{.}\PY{n}{mean}\PY{p}{(}\PY{p}{)}\PY{p}{)}\PY{p}{)}
\end{Verbatim}


    \begin{Verbatim}[commandchars=\\\{\}]
Aderiram em média 0.1195695564793657 usuários (resultado mais preciso)

    \end{Verbatim}

    \hypertarget{parei-aqui}{%
\paragraph{parei aqui}\label{parei-aqui}}

\begin{enumerate}
\def\labelenumi{\alph{enumi}.}
\setcounter{enumi}{4}
\tightlist
\item
  The number of times the \texttt{new\_page} and \texttt{treatment}
  don't line up.
\end{enumerate}

    \begin{Verbatim}[commandchars=\\\{\}]
{\color{incolor}In [{\color{incolor}3}]:} \PY{n}{a} \PY{o}{=} \PY{n}{dfdados}\PY{p}{[}\PY{p}{(}\PY{n}{dfdados}\PY{p}{[}\PY{l+s+s1}{\PYZsq{}}\PY{l+s+s1}{group}\PY{l+s+s1}{\PYZsq{}}\PY{p}{]} \PY{o}{==} \PY{l+s+s1}{\PYZsq{}}\PY{l+s+s1}{treatment}\PY{l+s+s1}{\PYZsq{}}\PY{p}{)} \PY{o}{!=} \PY{p}{(}\PY{n}{dfdados}\PY{p}{[}\PY{l+s+s1}{\PYZsq{}}\PY{l+s+s1}{landing\PYZus{}page}\PY{l+s+s1}{\PYZsq{}}\PY{p}{]} \PY{o}{==} \PY{l+s+s1}{\PYZsq{}}\PY{l+s+s1}{new\PYZus{}page}\PY{l+s+s1}{\PYZsq{}}\PY{p}{)}\PY{p}{]}
        \PY{n+nb}{print}\PY{p}{(}\PY{n}{a}\PY{o}{.}\PY{n}{head}\PY{p}{(}\PY{l+m+mi}{4}\PY{p}{)}\PY{p}{)}
        \PY{n}{a}\PY{o}{.}\PY{n}{shape}\PY{p}{[}\PY{l+m+mi}{0}\PY{p}{]}
\end{Verbatim}


    \begin{Verbatim}[commandchars=\\\{\}]
                          timestamp      group landing\_page  converted
user\_id                                                               
767017   2017-01-12 22:58:14.991443    control     new\_page          0
733976   2017-01-11 15:11:16.407599    control     new\_page          0
857184   2017-01-20 07:34:59.832626  treatment     old\_page          0
686623   2017-01-09 14:26:40.734775  treatment     old\_page          0

    \end{Verbatim}

\begin{Verbatim}[commandchars=\\\{\}]
{\color{outcolor}Out[{\color{outcolor}3}]:} 3893
\end{Verbatim}
            
    \begin{enumerate}
\def\labelenumi{\alph{enumi}.}
\setcounter{enumi}{5}
\tightlist
\item
  Do any of the rows have missing values?
\end{enumerate}

    \texttt{2.} For the rows where \textbf{treatment} is not aligned with
\textbf{new\_page} or \textbf{control} is not aligned with
\textbf{old\_page}, we cannot be sure if this row truly received the new
or old page. Use \textbf{Quiz 2} in the classroom to provide how we
should handle these rows.

\begin{enumerate}
\def\labelenumi{\alph{enumi}.}
\tightlist
\item
  Now use the answer to the quiz to create a new dataset that meets the
  specifications from the quiz. Store your new dataframe in
  \textbf{df2}.
\end{enumerate}

    \emph{Obs: já fiz isso lá em cima\ldots{} e também no projeto
anterior\ldots{} e aprendi que isso é bastante enganador! Diria:
\textbf{aparentemente} está tudo OK!, sem valores \textbf{NaN} pelo
menos\ldots{}}

    \begin{Verbatim}[commandchars=\\\{\}]
{\color{incolor}In [{\color{incolor}26}]:} \PY{n}{dfdados}\PY{o}{.}\PY{n}{info}\PY{p}{(}\PY{p}{)}
\end{Verbatim}


    \begin{Verbatim}[commandchars=\\\{\}]
<class 'pandas.core.frame.DataFrame'>
Int64Index: 294478 entries, 851104 to 715931
Data columns (total 4 columns):
timestamp       294478 non-null object
group           294478 non-null object
landing\_page    294478 non-null object
converted       294478 non-null int64
dtypes: int64(1), object(3)
memory usage: 11.2+ MB

    \end{Verbatim}

    \emph{Nesses casos, aprendi que usar o método \textbf{sample()} é melhor
do que \textbf{head()}\ldots{} posso rodar algumas vezes e pegar
amostras diferentes\ldots{}}

\emph{De onde eu tirei essa filtragem tão certinha? Foi só pular duas
questões para frente e fazer minha adaptação!}

    \begin{Verbatim}[commandchars=\\\{\}]
{\color{incolor}In [{\color{incolor}4}]:} \PY{c+c1}{\PYZsh{}dfaux1 = dfdados[(dfdados[\PYZsq{}group\PYZsq{}] == \PYZsq{}treatment\PYZsq{})]}
        \PY{c+c1}{\PYZsh{}print(dfaux1.sample(4))}
        \PY{c+c1}{\PYZsh{}dfaux2 = dfdados[(dfdados[\PYZsq{}landing\PYZus{}page\PYZsq{}] == \PYZsq{}new\PYZus{}page\PYZsq{})]}
        \PY{c+c1}{\PYZsh{}print(dfaux1.sample(4))}
        
        \PY{n}{df2} \PY{o}{=} \PY{n}{dfdados}\PY{p}{[}\PY{p}{(}\PY{n}{dfdados}\PY{p}{[}\PY{l+s+s1}{\PYZsq{}}\PY{l+s+s1}{group}\PY{l+s+s1}{\PYZsq{}}\PY{p}{]} \PY{o}{==} \PY{l+s+s1}{\PYZsq{}}\PY{l+s+s1}{treatment}\PY{l+s+s1}{\PYZsq{}}\PY{p}{)} \PY{o}{==} \PY{p}{(}\PY{n}{dfdados}\PY{p}{[}\PY{l+s+s1}{\PYZsq{}}\PY{l+s+s1}{landing\PYZus{}page}\PY{l+s+s1}{\PYZsq{}}\PY{p}{]} \PY{o}{==} \PY{l+s+s1}{\PYZsq{}}\PY{l+s+s1}{new\PYZus{}page}\PY{l+s+s1}{\PYZsq{}}\PY{p}{)}\PY{p}{]}
        \PY{n}{df2}\PY{o}{.}\PY{n}{sample}\PY{p}{(}\PY{l+m+mi}{10}\PY{p}{)}
\end{Verbatim}


\begin{Verbatim}[commandchars=\\\{\}]
{\color{outcolor}Out[{\color{outcolor}4}]:}                           timestamp      group landing\_page  converted
        user\_id                                                               
        820700   2017-01-07 03:45:03.766567    control     old\_page          0
        674677   2017-01-16 08:48:10.574350    control     old\_page          0
        764835   2017-01-20 02:35:11.726104    control     old\_page          0
        664579   2017-01-22 17:20:24.600227  treatment     new\_page          0
        726330   2017-01-07 09:13:39.903017    control     old\_page          0
        711230   2017-01-20 21:16:49.759821  treatment     new\_page          0
        930187   2017-01-09 11:09:02.354602  treatment     new\_page          0
        825080   2017-01-06 20:29:36.907516  treatment     new\_page          0
        765102   2017-01-20 11:16:12.462369  treatment     new\_page          0
        704329   2017-01-22 12:55:34.403475  treatment     new\_page          0
\end{Verbatim}
            
    Ou seja, estão todos os valores \textbf{alinhados}:

\emph{Eu vi um vídeo sobre \textbf{quarta normal} no Youtube e pelo que
entendi, seria exatamente esta consistência, dados de uma coluna que
dependem de valores de dados de outra coluna, num mesmo dataset.}

    \begin{Verbatim}[commandchars=\\\{\}]
{\color{incolor}In [{\color{incolor}5}]:} \PY{c+c1}{\PYZsh{} Double Check all of the correct rows were removed \PYZhy{} this should be 0}
        \PY{n}{df2}\PY{p}{[}\PY{p}{(}\PY{p}{(}\PY{n}{df2}\PY{p}{[}\PY{l+s+s1}{\PYZsq{}}\PY{l+s+s1}{group}\PY{l+s+s1}{\PYZsq{}}\PY{p}{]} \PY{o}{==} \PY{l+s+s1}{\PYZsq{}}\PY{l+s+s1}{treatment}\PY{l+s+s1}{\PYZsq{}}\PY{p}{)} \PY{o}{==} \PY{p}{(}\PY{n}{df2}\PY{p}{[}\PY{l+s+s1}{\PYZsq{}}\PY{l+s+s1}{landing\PYZus{}page}\PY{l+s+s1}{\PYZsq{}}\PY{p}{]} \PY{o}{==} \PY{l+s+s1}{\PYZsq{}}\PY{l+s+s1}{new\PYZus{}page}\PY{l+s+s1}{\PYZsq{}}\PY{p}{)}\PY{p}{)} \PY{o}{==} \PY{k+kc}{False}\PY{p}{]}\PY{o}{.}\PY{n}{shape}\PY{p}{[}\PY{l+m+mi}{0}\PY{p}{]}
\end{Verbatim}


\begin{Verbatim}[commandchars=\\\{\}]
{\color{outcolor}Out[{\color{outcolor}5}]:} 0
\end{Verbatim}
            
    \texttt{3.} Use \textbf{df2} and the cells below to answer questions for
\textbf{Quiz3} in the classroom.

    \begin{enumerate}
\def\labelenumi{\alph{enumi}.}
\tightlist
\item
  How many unique \textbf{user\_id}s are in \textbf{df2}?
\end{enumerate}

    \emph{Minha coluna user\_id foi definida como meu índice!}

\emph{Quando eu fiz isso, não atentei que poderia se tratar de um
dataset problemático, em que uma chave pudesse aparecer mais de uma
vez\ldots{} de qualquer maneira isso é um \textbf{defeito} e precisa ser
limpo!}

    \begin{Verbatim}[commandchars=\\\{\}]
{\color{incolor}In [{\color{incolor}6}]:} \PY{n+nb}{print}\PY{p}{(}\PY{l+s+s1}{\PYZsq{}}\PY{l+s+s1}{Existem }\PY{l+s+si}{\PYZob{}\PYZcb{}}\PY{l+s+s1}{ IDs de usuários únicos}\PY{l+s+s1}{\PYZsq{}}\PY{o}{.}\PY{n}{format}\PY{p}{(}\PY{n}{df2}\PY{o}{.}\PY{n}{index}\PY{o}{.}\PY{n}{nunique}\PY{p}{(}\PY{p}{)}\PY{p}{)}\PY{p}{)}
\end{Verbatim}


    \begin{Verbatim}[commandchars=\\\{\}]
Existem 290584 IDs de usuários únicos

    \end{Verbatim}

    \begin{enumerate}
\def\labelenumi{\alph{enumi}.}
\setcounter{enumi}{1}
\tightlist
\item
  There is one \textbf{user\_id} repeated in \textbf{df2}. What is it?
\end{enumerate}

    \emph{O valor emergiu como o primeiro na contagem!}

    \begin{Verbatim}[commandchars=\\\{\}]
{\color{incolor}In [{\color{incolor}6}]:} \PY{n}{contagem} \PY{o}{=} \PY{n}{df2}\PY{o}{.}\PY{n}{index}\PY{o}{.}\PY{n}{value\PYZus{}counts}\PY{p}{(}\PY{n}{sort}\PY{o}{=}\PY{k+kc}{True}\PY{p}{)}
        \PY{c+c1}{\PYZsh{}print(contagem)}
        \PY{n}{IDrepetido} \PY{o}{=} \PY{n}{contagem}\PY{o}{.}\PY{n}{index}\PY{p}{[}\PY{l+m+mi}{0}\PY{p}{]}
        
        \PY{n+nb}{print} \PY{p}{(}\PY{l+s+s1}{\PYZsq{}}\PY{l+s+s1}{O ID repetido :}\PY{l+s+s1}{\PYZsq{}}\PY{p}{,} \PY{n}{IDrepetido}\PY{p}{)}
        
        \PY{c+c1}{\PYZsh{}for indice,dado in contagem:}
        \PY{c+c1}{\PYZsh{}    if dado \PYZgt{} 1:}
        \PY{c+c1}{\PYZsh{}        print(indice)}
\end{Verbatim}


    \begin{Verbatim}[commandchars=\\\{\}]
O ID repetido : 773192

    \end{Verbatim}

    \hypertarget{section}{%
\subsubsection{3/5}\label{section}}

\begin{enumerate}
\def\labelenumi{\alph{enumi}.}
\setcounter{enumi}{2}
\tightlist
\item
  What is the row information for the repeat \textbf{user\_id}?
\end{enumerate}

    \emph{Invocando uma variável em filtro com @
\href{https://stackoverflow.com/questions/23974664/unable-to-query-a-local-variable-in-pandas-0-14-0}{aqui}}

    \begin{Verbatim}[commandchars=\\\{\}]
{\color{incolor}In [{\color{incolor}7}]:} \PY{c+c1}{\PYZsh{}print (IDrepetido)}
        
        \PY{n}{df2}\PY{o}{.}\PY{n}{query}\PY{p}{(}\PY{l+s+s1}{\PYZsq{}}\PY{l+s+s1}{index == @IDrepetido}\PY{l+s+s1}{\PYZsq{}}\PY{p}{)}
\end{Verbatim}


\begin{Verbatim}[commandchars=\\\{\}]
{\color{outcolor}Out[{\color{outcolor}7}]:}                           timestamp      group landing\_page  converted
        user\_id                                                               
        773192   2017-01-09 05:37:58.781806  treatment     new\_page          0
        773192   2017-01-14 02:55:59.590927  treatment     new\_page          0
\end{Verbatim}
            
    \begin{enumerate}
\def\labelenumi{\alph{enumi}.}
\setcounter{enumi}{3}
\tightlist
\item
  Remove \textbf{one} of the rows with a duplicate \textbf{user\_id},
  but keep your dataframe as \textbf{df2}.
\end{enumerate}

    \emph{Por que manter a última? Repare que as datas foram diferentes,
talvez o cara tenha ido e \textbf{retornado}! Me parece que a data mais
\textbf{atual} seria a melhor!}

\emph{Embora isso não tenha adiantado muito. O sujeito visitou duas
vezes a página mas não se ``converteu'', comprando nosso
produto\ldots{}}

\emph{Me enrolei um pouco, mas encontrei boas dicas
\href{https://stackoverflow.com/questions/13035764/remove-rows-with-duplicate-indices-pandas-dataframe-and-timeseries}{aqui}}

\emph{O operador lógico de \textbf{inversão \textasciitilde{}} indica
que eu quero todos, ao contrário, do registro duplicado. Eu copiei num
banco auxiliar para ter que evitar refazer as tarefas anteriores há cada
vez que fazia um teste}

\emph{Consegui este texto explicativo no Slack do curso: It is a unary
operator (taking a single argument) that is borrowed from C, where all
data types are just different ways of interpreting bytes. It is the
``invert'' or ``complement'' operation, in which all the bits of the
input data are reversed.}

    \begin{Verbatim}[commandchars=\\\{\}]
{\color{incolor}In [{\color{incolor}8}]:} \PY{n}{df3} \PY{o}{=} \PY{n}{df2}\PY{p}{[}\PY{o}{\PYZti{}}\PY{n}{df2}\PY{o}{.}\PY{n}{index}\PY{o}{.}\PY{n}{duplicated}\PY{p}{(}\PY{n}{keep}\PY{o}{=}\PY{l+s+s1}{\PYZsq{}}\PY{l+s+s1}{last}\PY{l+s+s1}{\PYZsq{}}\PY{p}{)}\PY{p}{]}
        
        \PY{c+c1}{\PYZsh{}df3 = df3[\PYZti{}df3.index.duplicated(keep=\PYZsq{}first\PYZsq{})]}
        \PY{c+c1}{\PYZsh{}df2[df2.index.duplicated(keep= \PYZsq{}last\PYZsq{}), inplace=True]}
        \PY{c+c1}{\PYZsh{}df2.drop\PYZus{}duplicates(subset=\PYZsq{}index\PYZsq{}, keep=\PYZsq{}last\PYZsq{}, inplace=True)}
        \PY{c+c1}{\PYZsh{}df2 = df2[df2.index.duplicated(keep=\PYZsq{}first\PYZsq{})]}
        
        \PY{n}{df3}\PY{o}{.}\PY{n}{query}\PY{p}{(}\PY{l+s+s1}{\PYZsq{}}\PY{l+s+s1}{index == 773192}\PY{l+s+s1}{\PYZsq{}}\PY{p}{)} \PY{c+c1}{\PYZsh{}teste no bandido}
\end{Verbatim}


\begin{Verbatim}[commandchars=\\\{\}]
{\color{outcolor}Out[{\color{outcolor}8}]:}                           timestamp      group landing\_page  converted
        user\_id                                                               
        773192   2017-01-14 02:55:59.590927  treatment     new\_page          0
\end{Verbatim}
            
    \emph{Copiando de volta ao lugar certo:}

    \begin{Verbatim}[commandchars=\\\{\}]
{\color{incolor}In [{\color{incolor}9}]:} \PY{n}{df2} \PY{o}{=} \PY{n}{df3}
        \PY{n}{df2}\PY{o}{.}\PY{n}{query}\PY{p}{(}\PY{l+s+s1}{\PYZsq{}}\PY{l+s+s1}{index == 773192}\PY{l+s+s1}{\PYZsq{}}\PY{p}{)} 
\end{Verbatim}


\begin{Verbatim}[commandchars=\\\{\}]
{\color{outcolor}Out[{\color{outcolor}9}]:}                           timestamp      group landing\_page  converted
        user\_id                                                               
        773192   2017-01-14 02:55:59.590927  treatment     new\_page          0
\end{Verbatim}
            
    \emph{Para que este código com if? Se eu quiser mais tarde fazer uma
função a partir disso, eu mudo o ``Passou'' para um}

\begin{verbatim}
return True
\end{verbatim}

    \begin{Verbatim}[commandchars=\\\{\}]
{\color{incolor}In [{\color{incolor}10}]:} \PY{n}{duplicatas} \PY{o}{=} \PY{n}{df3}\PY{o}{.}\PY{n}{index}\PY{o}{.}\PY{n}{value\PYZus{}counts}\PY{p}{(}\PY{p}{)}
         \PY{c+c1}{\PYZsh{}duplicatas}
         \PY{c+c1}{\PYZsh{}print(duplicatas)}
         \PY{c+c1}{\PYZsh{}print(len(duplicatas))}
         
         \PY{n+nb}{print}\PY{p}{(}\PY{l+s+s1}{\PYZsq{}}\PY{l+s+s1}{Verificando duplicatas :}\PY{l+s+s1}{\PYZsq{}}\PY{p}{,} \PY{n+nb}{len}\PY{p}{(}\PY{n}{duplicatas}\PY{p}{[}\PY{n}{duplicatas} \PY{o}{\PYZgt{}} \PY{l+m+mi}{1}\PY{p}{]}\PY{p}{)}\PY{p}{)}
         
         \PY{k}{if} \PY{n+nb}{len}\PY{p}{(}\PY{n}{duplicatas}\PY{p}{[}\PY{n}{duplicatas} \PY{o}{\PYZgt{}} \PY{l+m+mi}{1}\PY{p}{]}\PY{p}{)} \PY{o}{==} \PY{l+m+mi}{0}\PY{p}{:}
             \PY{n+nb}{print}\PY{p}{(}\PY{l+s+s1}{\PYZsq{}}\PY{l+s+s1}{Passou}\PY{l+s+s1}{\PYZsq{}}\PY{p}{)}
\end{Verbatim}


    \begin{Verbatim}[commandchars=\\\{\}]
Verificando duplicatas : 0
Passou

    \end{Verbatim}

    \hypertarget{parei-aqui}{%
\paragraph{Parei aqui}\label{parei-aqui}}

    \texttt{4.} Use \textbf{df2} in the below cells to answer the quiz
questions related to \textbf{Quiz 4} in the classroom.

\begin{enumerate}
\def\labelenumi{\alph{enumi}.}
\tightlist
\item
  What is the probability of an individual converting regardless of the
  page they receive?
\end{enumerate}

    \begin{Verbatim}[commandchars=\\\{\}]
{\color{incolor}In [{\color{incolor}65}]:} \PY{c+c1}{\PYZsh{}df2.head(4)}
         \PY{n+nb}{print} \PY{p}{(}\PY{l+s+s1}{\PYZsq{}}\PY{l+s+s1}{Média de conversões :}\PY{l+s+s1}{\PYZsq{}}\PY{p}{,} \PY{n}{df2}\PY{o}{.}\PY{n}{converted}\PY{o}{.}\PY{n}{mean}\PY{p}{(}\PY{p}{)}\PY{p}{)}
\end{Verbatim}


    \begin{Verbatim}[commandchars=\\\{\}]
Média de conversões : 0.11959667567149027

    \end{Verbatim}

    \begin{enumerate}
\def\labelenumi{\alph{enumi}.}
\setcounter{enumi}{1}
\tightlist
\item
  Given that an individual was in the \texttt{control} group, what is
  the probability they converted?
\end{enumerate}

    \emph{Sim, é um pouco mais alto\ldots{} e me parece bastante
\textbf{pouca} a diferença\ldots{} mas na direção \textbf{errada}!}

    \begin{Verbatim}[commandchars=\\\{\}]
{\color{incolor}In [{\color{incolor}11}]:} \PY{n}{dfcontrole} \PY{o}{=} \PY{n}{df2}\PY{o}{.}\PY{n}{query}\PY{p}{(}\PY{l+s+s1}{\PYZsq{}}\PY{l+s+s1}{group == }\PY{l+s+s1}{\PYZdq{}}\PY{l+s+s1}{control}\PY{l+s+s1}{\PYZdq{}}\PY{l+s+s1}{\PYZsq{}}\PY{p}{)}
         \PY{n+nb}{print}\PY{p}{(}\PY{n}{dfcontrole}\PY{o}{.}\PY{n}{head}\PY{p}{(}\PY{l+m+mi}{4}\PY{p}{)}\PY{p}{)}
         \PY{n+nb}{print} \PY{p}{(}\PY{l+s+s1}{\PYZsq{}}\PY{l+s+s1}{Média de conversões no Controle :}\PY{l+s+s1}{\PYZsq{}}\PY{p}{,} \PY{n}{dfcontrole}\PY{o}{.}\PY{n}{converted}\PY{o}{.}\PY{n}{mean}\PY{p}{(}\PY{p}{)}\PY{p}{)}
\end{Verbatim}


    \begin{Verbatim}[commandchars=\\\{\}]
                          timestamp    group landing\_page  converted
user\_id                                                             
851104   2017-01-21 22:11:48.556739  control     old\_page          0
804228   2017-01-12 08:01:45.159739  control     old\_page          0
864975   2017-01-21 01:52:26.210827  control     old\_page          1
936923   2017-01-10 15:20:49.083499  control     old\_page          0
Média de conversões no Controle : 0.1203863045004612

    \end{Verbatim}

    \begin{enumerate}
\def\labelenumi{\alph{enumi}.}
\setcounter{enumi}{2}
\tightlist
\item
  Given that an individual was in the \texttt{treatment} group, what is
  the probability they converted?
\end{enumerate}

    \emph{Desculpe-me, sou como Lázaro\ldots{} é necessário \textbf{ver}
para \textbf{crer}\ldots{}}

    \begin{Verbatim}[commandchars=\\\{\}]
{\color{incolor}In [{\color{incolor}12}]:} \PY{n}{dftratamento} \PY{o}{=} \PY{n}{df2}\PY{o}{.}\PY{n}{query}\PY{p}{(}\PY{l+s+s1}{\PYZsq{}}\PY{l+s+s1}{group == }\PY{l+s+s1}{\PYZdq{}}\PY{l+s+s1}{treatment}\PY{l+s+s1}{\PYZdq{}}\PY{l+s+s1}{\PYZsq{}}\PY{p}{)}
         \PY{n+nb}{print}\PY{p}{(}\PY{n}{dftratamento}\PY{o}{.}\PY{n}{head}\PY{p}{(}\PY{l+m+mi}{4}\PY{p}{)}\PY{p}{)}
         \PY{n+nb}{print} \PY{p}{(}\PY{l+s+s1}{\PYZsq{}}\PY{l+s+s1}{Média de conversões no Tratamento :}\PY{l+s+s1}{\PYZsq{}}\PY{p}{,} \PY{n}{dftratamento}\PY{o}{.}\PY{n}{converted}\PY{o}{.}\PY{n}{mean}\PY{p}{(}\PY{p}{)}\PY{p}{)}
\end{Verbatim}


    \begin{Verbatim}[commandchars=\\\{\}]
                          timestamp      group landing\_page  converted
user\_id                                                               
661590   2017-01-11 16:55:06.154213  treatment     new\_page          0
853541   2017-01-08 18:28:03.143765  treatment     new\_page          0
679687   2017-01-19 03:26:46.940749  treatment     new\_page          1
817355   2017-01-04 17:58:08.979471  treatment     new\_page          1
Média de conversões no Tratamento : 0.11880806551510564

    \end{Verbatim}

    \begin{enumerate}
\def\labelenumi{\alph{enumi}.}
\setcounter{enumi}{3}
\tightlist
\item
  What is the probability that an individual received the new page?
\end{enumerate}

    \emph{Foi tipo: \textbf{metade-metade}\ldots{}}

    \begin{Verbatim}[commandchars=\\\{\}]
{\color{incolor}In [{\color{incolor}80}]:} \PY{n+nb}{print} \PY{p}{(}\PY{l+s+s1}{\PYZsq{}}\PY{l+s+s1}{Probabilidade de ser direcionado à páguna nova :}\PY{l+s+s1}{\PYZsq{}}\PY{p}{,} \PY{p}{(}\PY{n}{df2}\PY{o}{.}\PY{n}{landing\PYZus{}page} \PY{o}{==} \PY{l+s+s2}{\PYZdq{}}\PY{l+s+s2}{new\PYZus{}page}\PY{l+s+s2}{\PYZdq{}}\PY{p}{)}\PY{o}{.}\PY{n}{mean}\PY{p}{(}\PY{p}{)}\PY{p}{)}
\end{Verbatim}


    \begin{Verbatim}[commandchars=\\\{\}]
Probabilidade de ser direcionado à páguna nova : 0.5000636646764286

    \end{Verbatim}

    \begin{enumerate}
\def\labelenumi{\alph{enumi}.}
\setcounter{enumi}{4}
\tightlist
\item
  Consider your results from a. through d. above, and explain below
  whether you think there is sufficient evidence to say that the new
  treatment page leads to more conversions.
\end{enumerate}

    \textbf{Para mim parece não haver forte indicativo (indícios) de que a
página nova realmente produziu mais conversões do que a antiga\ldots{}
os resultados me parecem próximos demais\ldots{}}

     \#\#\# Part II - A/B Test

Notice that because of the time stamp associated with each event, you
could technically run a hypothesis test continuously as each observation
was observed.

However, then the hard question is do you stop as soon as one page is
considered significantly better than another or does it need to happen
consistently for a certain amount of time? How long do you run to render
a decision that neither page is better than another?

These questions are the difficult parts associated with A/B tests in
general.

\texttt{1.} For now, consider you need to make the decision just based
on all the data provided. If you want to assume that the old page is
better unless the new page proves to be definitely better at a Type I
error rate of 5\%, what should your null and alternative hypotheses be?
You can state your hypothesis in terms of words or in terms of
\textbf{\(p_{old}\)} and \textbf{\(p_{new}\)}, which are the converted
rates for the old and new pages.

    \(Hipótese_{zero} \rightarrow H_0: p_{new} - p_{old} \leq 0\) (a página
\textbf{antiga} é tão boa ou melhor do que a nova para ``converter'')

\(Hipótese_{alternativa} \rightarrow H_1: p_{new} - p_{old} > 0\) (a
página \textbf{nova} é melhor para ``converter'' pessoas)

\begin{itemize}
\tightlist
\item
  Margem de confiança para erro \textbf{Tipo I} (falso positivo):
  \textbf{5\%}
\end{itemize}

\emph{Nada como o bom \textbf{codificador LaTex} online
\href{https://latex.codecogs.com/eqneditor/editor.php}{aqui}}

    \texttt{2.} Assume under the null hypothesis, \(p_{new}\) and
\(p_{old}\) both have ``true'' success rates equal to the
\textbf{converted} success rate regardless of page - that is \(p_{new}\)
and \(p_{old}\) are equal. Furthermore, assume they are equal to the
\textbf{converted} rate in \textbf{ab\_data.csv} regardless of the page.

Use a sample size for each page equal to the ones in
\textbf{ab\_data.csv}.

Perform the sampling distribution for the difference in
\textbf{converted} between the two pages over 10,000 iterations of
calculating an estimate from the null.

Use the cells below to provide the necessary parts of this simulation.
If this doesn't make complete sense right now, don't worry - you are
going to work through the problems below to complete this problem. You
can use \textbf{Quiz 5} in the classroom to make sure you are on the
right track.

    \hypertarget{parei-aqui}{%
\paragraph{Parei aqui}\label{parei-aqui}}

    \begin{enumerate}
\def\labelenumi{\alph{enumi}.}
\tightlist
\item
  What is the \textbf{convert rate} for \(p_{new}\) under the null?
\end{enumerate}

    \emph{Qual é a condição \textbf{limite} de validade da \(H_0\)? É que a
taxa de ``conversões'' da nova página é \textbf{igual} à da página
anterior (ou seja, que a página nova será tão boa quanto a anterior).
Então a \(p_{nova}\) será igual à \$p\_\{antiga\}. Assim, como não faz
diferença se estamos na página nova ou antiga\ldots{}}

    \begin{Verbatim}[commandchars=\\\{\}]
{\color{incolor}In [{\color{incolor}13}]:} \PY{n}{dfpnova} \PY{o}{=} \PY{n}{df2}\PY{p}{[}\PY{l+s+s1}{\PYZsq{}}\PY{l+s+s1}{converted}\PY{l+s+s1}{\PYZsq{}}\PY{p}{]}
         \PY{n+nb}{print} \PY{p}{(}\PY{l+s+s1}{\PYZsq{}}\PY{l+s+s1}{Segundo a H0, a taxa de conversão para a página nova será :}\PY{l+s+s1}{\PYZsq{}}\PY{p}{,} \PY{n}{dfpnova}\PY{o}{.}\PY{n}{mean}\PY{p}{(}\PY{p}{)}\PY{p}{)}
\end{Verbatim}


    \begin{Verbatim}[commandchars=\\\{\}]
Segundo a H0, a taxa de conversão para a página nova será : 0.11959708724499628

    \end{Verbatim}

    \begin{enumerate}
\def\labelenumi{\alph{enumi}.}
\setcounter{enumi}{1}
\tightlist
\item
  What is the \textbf{convert rate} for \(p_{old}\) under the null? 
\end{enumerate}

    \begin{Verbatim}[commandchars=\\\{\}]
{\color{incolor}In [{\color{incolor}14}]:} \PY{n}{dfpvelha} \PY{o}{=} \PY{n}{dfpnova}
         \PY{n+nb}{print} \PY{p}{(}\PY{l+s+s1}{\PYZsq{}}\PY{l+s+s1}{Segundo a H0, a taxa de conversão para a página antiga será :}\PY{l+s+s1}{\PYZsq{}}\PY{p}{,} \PY{n}{dfpvelha}\PY{o}{.}\PY{n}{mean}\PY{p}{(}\PY{p}{)}\PY{p}{)}
\end{Verbatim}


    \begin{Verbatim}[commandchars=\\\{\}]
Segundo a H0, a taxa de conversão para a página antiga será : 0.11959708724499628

    \end{Verbatim}

    \hypertarget{section}{%
\subsubsection{4/5}\label{section}}

\begin{enumerate}
\def\labelenumi{\alph{enumi}.}
\setcounter{enumi}{2}
\tightlist
\item
  What is \(n_{new}\)?
\end{enumerate}

    \begin{Verbatim}[commandchars=\\\{\}]
{\color{incolor}In [{\color{incolor}17}]:} \PY{n+nb}{print}\PY{p}{(}\PY{n}{df2}\PY{o}{.}\PY{n}{sample}\PY{p}{(}\PY{l+m+mi}{4}\PY{p}{)}\PY{p}{)}
         \PY{n}{dfnovapagina} \PY{o}{=} \PY{n}{df2}\PY{o}{.}\PY{n}{query}\PY{p}{(}\PY{l+s+s1}{\PYZsq{}}\PY{l+s+s1}{landing\PYZus{}page == }\PY{l+s+s1}{\PYZdq{}}\PY{l+s+s1}{new\PYZus{}page}\PY{l+s+s1}{\PYZdq{}}\PY{l+s+s1}{\PYZsq{}}\PY{p}{)}
         \PY{n+nb}{print}\PY{p}{(}\PY{n}{dfnovapagina}\PY{o}{.}\PY{n}{sample}\PY{p}{(}\PY{l+m+mi}{2}\PY{p}{)}\PY{p}{)}
         \PY{n+nb}{print}\PY{p}{(}\PY{l+s+s1}{\PYZsq{}}\PY{l+s+s1}{O número de acessos à nova página foi :}\PY{l+s+s1}{\PYZsq{}}\PY{p}{,} \PY{n+nb}{len}\PY{p}{(}\PY{n}{dfnovapagina}\PY{p}{)}\PY{p}{)}
\end{Verbatim}


    \begin{Verbatim}[commandchars=\\\{\}]
                          timestamp      group landing\_page  converted
user\_id                                                               
896507   2017-01-07 01:14:35.009907    control     old\_page          1
897948   2017-01-22 07:11:17.635813    control     old\_page          1
658910   2017-01-07 07:37:06.500306  treatment     new\_page          0
650570   2017-01-12 06:58:28.470289  treatment     new\_page          0
                          timestamp      group landing\_page  converted
user\_id                                                               
778872   2017-01-08 15:39:37.019025  treatment     new\_page          0
901156   2017-01-19 23:53:41.448477  treatment     new\_page          0
O número de acessos à nova página foi : 145310

    \end{Verbatim}

    \begin{enumerate}
\def\labelenumi{\alph{enumi}.}
\setcounter{enumi}{3}
\tightlist
\item
  What is \(n_{old}\)?
\end{enumerate}

    \begin{Verbatim}[commandchars=\\\{\}]
{\color{incolor}In [{\color{incolor}18}]:} \PY{n}{dfvelhapagina} \PY{o}{=} \PY{n}{df2}\PY{o}{.}\PY{n}{query}\PY{p}{(}\PY{l+s+s1}{\PYZsq{}}\PY{l+s+s1}{landing\PYZus{}page == }\PY{l+s+s1}{\PYZdq{}}\PY{l+s+s1}{old\PYZus{}page}\PY{l+s+s1}{\PYZdq{}}\PY{l+s+s1}{\PYZsq{}}\PY{p}{)}
         \PY{n+nb}{print}\PY{p}{(}\PY{n}{dfvelhapagina}\PY{o}{.}\PY{n}{sample}\PY{p}{(}\PY{l+m+mi}{2}\PY{p}{)}\PY{p}{)}
         \PY{n+nb}{print}\PY{p}{(}\PY{l+s+s1}{\PYZsq{}}\PY{l+s+s1}{O número de acessos à nova página foi :}\PY{l+s+s1}{\PYZsq{}}\PY{p}{,} \PY{n+nb}{len}\PY{p}{(}\PY{n}{dfvelhapagina}\PY{p}{)}\PY{p}{)}
\end{Verbatim}


    \begin{Verbatim}[commandchars=\\\{\}]
                          timestamp    group landing\_page  converted
user\_id                                                             
789160   2017-01-13 12:01:50.700659  control     old\_page          1
874791   2017-01-22 22:11:24.799994  control     old\_page          0
O número de acessos à nova página foi : 145274

    \end{Verbatim}

    \hypertarget{parei-aqui}{%
\paragraph{Parei aqui}\label{parei-aqui}}

    \begin{enumerate}
\def\labelenumi{\alph{enumi}.}
\setcounter{enumi}{4}
\tightlist
\item
  Simulate \(n_{new}\) transactions with a convert rate of \(p_{new}\)
  under the null. Store these \(n_{new}\) 1's and 0's in
  \textbf{new\_page\_converted}.
\end{enumerate}

    A ideia aqui é a seguinte. Eu tenho uma \textbf{estatística}, ou seja:

\begin{itemize}
\item
  uma média
\item
  uma quantidade n de elementos
\end{itemize}

Supondo que eu não tivesse mais aquele dataset original. Eu posso
reconstruir um similar, com suas mesmas características
\textbf{estatísticas}. Em Física, chamaríamos isso de formatar um modelo
\textbf{granular} a partir de determinadas equações ou condições de
existência.

    \begin{Verbatim}[commandchars=\\\{\}]
{\color{incolor}In [{\color{incolor}19}]:} \PY{k+kn}{import} \PY{n+nn}{numpy} \PY{k}{as} \PY{n+nn}{np}
         
         \PY{n}{a} \PY{o}{=} \PY{p}{[}\PY{l+m+mi}{0}\PY{p}{,} \PY{l+m+mi}{1}\PY{p}{]} \PY{c+c1}{\PYZsh{}a minha semente de opções}
         \PY{n}{nnova} \PY{o}{=} \PY{n+nb}{len}\PY{p}{(}\PY{n}{dfnovapagina}\PY{p}{)}
         \PY{c+c1}{\PYZsh{}print(nnova)}
         \PY{n}{pnova} \PY{o}{=} \PY{n}{dfpnova}\PY{o}{.}\PY{n}{mean}\PY{p}{(}\PY{p}{)}
         
         \PY{n}{new\PYZus{}page\PYZus{}converted} \PY{o}{=} \PY{n}{np}\PY{o}{.}\PY{n}{random}\PY{o}{.}\PY{n}{choice}\PY{p}{(}\PY{n}{a}\PY{p}{,} \PY{n}{size}\PY{o}{=}\PY{n}{nnova}\PY{p}{,} \PY{n}{p}\PY{o}{=}\PY{p}{[}\PY{l+m+mi}{1}\PY{o}{\PYZhy{}}\PY{n}{pnova}\PY{p}{,} \PY{n}{pnova}\PY{p}{]}\PY{p}{)}
         \PY{n+nb}{print}\PY{p}{(}\PY{l+s+s1}{\PYZsq{}}\PY{l+s+s1}{Nova \PYZhy{} meu tamanho amostral original :}\PY{l+s+si}{\PYZob{}\PYZcb{}}\PY{l+s+s1}{, reconstuído :}\PY{l+s+si}{\PYZob{}\PYZcb{}}\PY{l+s+s1}{\PYZsq{}}\PY{o}{.}\PY{n}{format}\PY{p}{(}\PY{n}{nnova}\PY{p}{,}
                                                                                  \PY{n+nb}{len}\PY{p}{(}\PY{n}{new\PYZus{}page\PYZus{}converted}\PY{p}{)}\PY{p}{)}\PY{p}{)}
\end{Verbatim}


    \begin{Verbatim}[commandchars=\\\{\}]
Nova - meu tamanho amostral original :145310, reconstuído :145310

    \end{Verbatim}

    \begin{enumerate}
\def\labelenumi{\alph{enumi}.}
\setcounter{enumi}{5}
\tightlist
\item
  Simulate \(n_{old}\) transactions with a convert rate of \(p_{old}\)
  under the null. Store these \(n_{old}\) 1's and 0's in
  \textbf{old\_page\_converted}.
\end{enumerate}

    \begin{Verbatim}[commandchars=\\\{\}]
{\color{incolor}In [{\color{incolor}20}]:} \PY{n}{nvelha} \PY{o}{=} \PY{n+nb}{len}\PY{p}{(}\PY{n}{dfvelhapagina}\PY{p}{)}
         \PY{n}{pvelha} \PY{o}{=} \PY{n}{dfpvelha}\PY{o}{.}\PY{n}{mean}\PY{p}{(}\PY{p}{)}
         
         \PY{n}{old\PYZus{}page\PYZus{}converted} \PY{o}{=} \PY{n}{np}\PY{o}{.}\PY{n}{random}\PY{o}{.}\PY{n}{choice}\PY{p}{(}\PY{n}{a}\PY{p}{,} \PY{n}{size}\PY{o}{=}\PY{n}{nvelha}\PY{p}{,} \PY{n}{p}\PY{o}{=}\PY{p}{[}\PY{l+m+mi}{1}\PY{o}{\PYZhy{}}\PY{n}{pvelha}\PY{p}{,} \PY{n}{pvelha}\PY{p}{]}\PY{p}{)}
         \PY{n+nb}{print}\PY{p}{(}\PY{l+s+s1}{\PYZsq{}}\PY{l+s+s1}{Velha \PYZhy{} meu tamanho amostral original:}\PY{l+s+si}{\PYZob{}\PYZcb{}}\PY{l+s+s1}{, reconstuído:}\PY{l+s+si}{\PYZob{}\PYZcb{}}\PY{l+s+s1}{\PYZsq{}}\PY{o}{.}\PY{n}{format}\PY{p}{(}\PY{n}{nvelha}\PY{p}{,}\PY{n+nb}{len}\PY{p}{(}\PY{n}{old\PYZus{}page\PYZus{}converted}\PY{p}{)}\PY{p}{)}\PY{p}{)}
\end{Verbatim}


    \begin{Verbatim}[commandchars=\\\{\}]
Velha - meu tamanho amostral original:145274, reconstuído:145274

    \end{Verbatim}

    \begin{enumerate}
\def\labelenumi{\alph{enumi}.}
\setcounter{enumi}{6}
\tightlist
\item
  Find \(p_{new}\) - \(p_{old}\) for your simulated values from part (e)
  and (f).
\end{enumerate}

    \emph{Observe que a simulação gerou uma diferença realmente
\textbf{muito pequena}. Como estamos com o \textbf{random choice}
aplicado, esta sempre se manterá igual. Mas se o desabilitarmos,
poderíamos ver que esse número varia um pouco, podendo eventualmente
\textbf{zerar}. Isso pode ser chamado de \textbf{ruído}. É mais ou menos
assim\ldots{}}

\emph{Eu não tenho certeza de onde este exercício irá levar, mas no
curso (\textbf{M3, Aula 14}) o instrutor fala: ``\ldots{} \textbf{quanto
mais coisas você testa, mais provável que você observe diferenças
significativas, apenas pelo acaso. Isso acontece quando fazemos
avaliações de várias métricas ao mesmo tempo. A probabilidade de falso
positivo aumenta conforme você aumenta o número de métricas}\ldots{}''.
Bom, o mesmo acontece quando você aumenta demais o número de
amostragens\ldots{} o ruído começa a ser percebido pelo modelo como
\textbf{significativo}, quando de fato náo é!}

    \begin{Verbatim}[commandchars=\\\{\}]
{\color{incolor}In [{\color{incolor}22}]:} \PY{n}{diff} \PY{o}{=} \PY{n}{new\PYZus{}page\PYZus{}converted}\PY{o}{.}\PY{n}{mean}\PY{p}{(}\PY{p}{)} \PY{o}{\PYZhy{}} \PY{n}{old\PYZus{}page\PYZus{}converted}\PY{o}{.}\PY{n}{mean}\PY{p}{(}\PY{p}{)}
         \PY{n+nb}{print}\PY{p}{(}\PY{l+s+s1}{\PYZsq{}}\PY{l+s+s1}{A diferença de probabilidades simuladas para minha nova e velha página é :}\PY{l+s+s1}{\PYZsq{}}\PY{p}{,} \PY{n}{diff}\PY{p}{)}
\end{Verbatim}


    \begin{Verbatim}[commandchars=\\\{\}]
A diferença de probabilidades simuladas para minha nova e velha página é : 0.000995769969952126

    \end{Verbatim}

    \begin{enumerate}
\def\labelenumi{\alph{enumi}.}
\setcounter{enumi}{7}
\tightlist
\item
  Simulate 10,000 \(p_{new}\) - \(p_{old}\) values using this same
  process similarly to the one you calculated in parts \textbf{a.
  through g.} above. Store all 10,000 values in a numpy array called
  \textbf{p\_diffs}.
\end{enumerate}

    \emph{Cuidado: isso demora muito para rodar!}

    \begin{Verbatim}[commandchars=\\\{\}]
{\color{incolor}In [{\color{incolor}23}]:} \PY{k}{def} \PY{n+nf}{embaralhador}\PY{p}{(}\PY{n}{nnova}\PY{p}{,} \PY{n}{nvelha}\PY{p}{,} \PY{n}{pnova}\PY{p}{,} \PY{n}{pvelha}\PY{p}{,} \PY{n}{tamanho}\PY{p}{,} \PY{n}{a}\PY{o}{=}\PY{p}{[}\PY{l+m+mi}{0}\PY{p}{,}\PY{l+m+mi}{1}\PY{p}{]}\PY{p}{)}\PY{p}{:}
             \PY{k+kn}{import} \PY{n+nn}{numpy} \PY{k}{as} \PY{n+nn}{np}
             \PY{n}{captadiff} \PY{o}{=} \PY{p}{[}\PY{p}{]}
         
             \PY{k}{for} \PY{n}{i} \PY{o+ow}{in} \PY{n+nb}{range}\PY{p}{(}\PY{n}{tamanho}\PY{p}{)}\PY{p}{:}
                 \PY{n}{new\PYZus{}sample} \PY{o}{=} \PY{n}{np}\PY{o}{.}\PY{n}{random}\PY{o}{.}\PY{n}{choice}\PY{p}{(}\PY{n}{a}\PY{p}{,} \PY{n}{size}\PY{o}{=}\PY{n}{nnova}\PY{p}{,} \PY{n}{p}\PY{o}{=}\PY{p}{[}\PY{l+m+mi}{1}\PY{o}{\PYZhy{}}\PY{n}{pnova}\PY{p}{,} \PY{n}{pnova}\PY{p}{]}\PY{p}{)}
                 \PY{n}{old\PYZus{}sample} \PY{o}{=} \PY{n}{np}\PY{o}{.}\PY{n}{random}\PY{o}{.}\PY{n}{choice}\PY{p}{(}\PY{n}{a}\PY{p}{,} \PY{n}{size}\PY{o}{=}\PY{n}{nvelha}\PY{p}{,} \PY{n}{p}\PY{o}{=}\PY{p}{[}\PY{l+m+mi}{1}\PY{o}{\PYZhy{}}\PY{n}{pvelha}\PY{p}{,} \PY{n}{pvelha}\PY{p}{]}\PY{p}{)}
                 \PY{n}{captadiff}\PY{o}{.}\PY{n}{append}\PY{p}{(}\PY{n}{new\PYZus{}sample}\PY{o}{.}\PY{n}{mean}\PY{p}{(}\PY{p}{)} \PY{o}{\PYZhy{}} \PY{n}{old\PYZus{}sample}\PY{o}{.}\PY{n}{mean}\PY{p}{(}\PY{p}{)}\PY{p}{)}
         
             \PY{k}{return} \PY{n}{np}\PY{o}{.}\PY{n}{array}\PY{p}{(}\PY{n}{captadiff}\PY{p}{)}
             
         \PY{n}{nnova} \PY{o}{=} \PY{n+nb}{len}\PY{p}{(}\PY{n}{dfnovapagina}\PY{p}{)}
         \PY{n}{pnova} \PY{o}{=} \PY{n}{dfpnova}\PY{o}{.}\PY{n}{mean}\PY{p}{(}\PY{p}{)}
         \PY{n}{nvelha} \PY{o}{=} \PY{n+nb}{len}\PY{p}{(}\PY{n}{dfvelhapagina}\PY{p}{)}
         \PY{n}{pvelha} \PY{o}{=} \PY{n}{dfpvelha}\PY{o}{.}\PY{n}{mean}\PY{p}{(}\PY{p}{)}
             
         \PY{n}{p\PYZus{}diffs} \PY{o}{=} \PY{n}{embaralhador}\PY{p}{(}\PY{n}{nnova}\PY{p}{,} \PY{n}{nvelha}\PY{p}{,} \PY{n}{pnova}\PY{p}{,} \PY{n}{pvelha}\PY{p}{,} \PY{l+m+mi}{10000}\PY{p}{)}
\end{Verbatim}


    Pareceu OK:

\begin{itemize}
\tightlist
\item
  gerou uma lista com 10.000 posições em ponto flutuante, representando
  as diferenças
\end{itemize}

    \begin{Verbatim}[commandchars=\\\{\}]
{\color{incolor}In [{\color{incolor}132}]:} \PY{n+nb}{print}\PY{p}{(}\PY{n}{p\PYZus{}diffs}\PY{p}{)}
          \PY{n+nb}{print}\PY{p}{(}\PY{n+nb}{len}\PY{p}{(}\PY{n}{p\PYZus{}diffs}\PY{p}{)}\PY{p}{)}
          \PY{n}{p\PYZus{}diffs}\PY{p}{[}\PY{l+m+mi}{0}\PY{p}{]}
\end{Verbatim}


    \begin{Verbatim}[commandchars=\\\{\}]
[-0.00065689 -0.00066359  0.0013392  {\ldots} -0.00073949  0.00032063
 -0.00077361]
10000

    \end{Verbatim}

\begin{Verbatim}[commandchars=\\\{\}]
{\color{outcolor}Out[{\color{outcolor}132}]:} -0.0006568860245934893
\end{Verbatim}
            
    \begin{Verbatim}[commandchars=\\\{\}]
{\color{incolor}In [{\color{incolor}24}]:} \PY{n}{np}\PY{o}{.}\PY{n}{save}\PY{p}{(}\PY{l+s+s1}{\PYZsq{}}\PY{l+s+s1}{matrizdiff.npy}\PY{l+s+s1}{\PYZsq{}}\PY{p}{,} \PY{n}{p\PYZus{}diffs}\PY{p}{)}
\end{Verbatim}


    \begin{Verbatim}[commandchars=\\\{\}]
{\color{incolor}In [{\color{incolor}11}]:} \PY{n}{df2}\PY{o}{.}\PY{n}{to\PYZus{}csv}\PY{p}{(}\PY{l+s+s1}{\PYZsq{}}\PY{l+s+s1}{abdf2.csv}\PY{l+s+s1}{\PYZsq{}}\PY{p}{,} \PY{n}{sep}\PY{o}{=}\PY{l+s+s1}{\PYZsq{}}\PY{l+s+se}{\PYZbs{}t}\PY{l+s+s1}{\PYZsq{}}\PY{p}{,} \PY{n}{encoding}\PY{o}{=}\PY{l+s+s1}{\PYZsq{}}\PY{l+s+s1}{utf\PYZhy{}8}\PY{l+s+s1}{\PYZsq{}}\PY{p}{,} \PY{n}{index}\PY{o}{=}\PY{l+s+s1}{\PYZsq{}}\PY{l+s+s1}{usr\PYZus{}id}\PY{l+s+s1}{\PYZsq{}}\PY{p}{)}
\end{Verbatim}


    \hypertarget{parei-aqui}{%
\paragraph{Parei aqui}\label{parei-aqui}}

    Truque do preguiçoso: busquei na documentação do numpy como gravar um
binário com os valores gerados para continuar mais tarde\ldots{}

    \begin{Verbatim}[commandchars=\\\{\}]
{\color{incolor}In [{\color{incolor}26}]:} \PY{k+kn}{import} \PY{n+nn}{numpy} \PY{k}{as} \PY{n+nn}{np}
\end{Verbatim}


    \begin{Verbatim}[commandchars=\\\{\}]
{\color{incolor}In [{\color{incolor}9}]:} \PY{n}{p\PYZus{}diffs} \PY{o}{=} \PY{n}{np}\PY{o}{.}\PY{n}{load}\PY{p}{(}\PY{l+s+s1}{\PYZsq{}}\PY{l+s+s1}{matrizdiff.npy}\PY{l+s+s1}{\PYZsq{}}\PY{p}{)}
\end{Verbatim}


    \begin{Verbatim}[commandchars=\\\{\}]
{\color{incolor}In [{\color{incolor}10}]:} \PY{n}{p\PYZus{}diffs}
\end{Verbatim}


\begin{Verbatim}[commandchars=\\\{\}]
{\color{outcolor}Out[{\color{outcolor}10}]:} array([-0.00065689, -0.00066359,  0.0013392 , {\ldots}, -0.00073949,
                 0.00032063, -0.00077361])
\end{Verbatim}
            
    \begin{Verbatim}[commandchars=\\\{\}]
{\color{incolor}In [{\color{incolor}56}]:} \PY{n}{df2} \PY{o}{=} \PY{n}{pd}\PY{o}{.}\PY{n}{read\PYZus{}csv}\PY{p}{(}\PY{l+s+s1}{\PYZsq{}}\PY{l+s+s1}{abdf2.csv}\PY{l+s+s1}{\PYZsq{}}\PY{p}{,} \PY{n}{sep}\PY{o}{=}\PY{l+s+s1}{\PYZsq{}}\PY{l+s+se}{\PYZbs{}t}\PY{l+s+s1}{\PYZsq{}}\PY{p}{,} \PY{n}{encoding}\PY{o}{=}\PY{l+s+s1}{\PYZsq{}}\PY{l+s+s1}{utf\PYZhy{}8}\PY{l+s+s1}{\PYZsq{}}\PY{p}{,} \PY{n}{index\PYZus{}col}\PY{o}{=}\PY{k+kc}{False}\PY{p}{)}
         \PY{n}{df2}\PY{o}{.}\PY{n}{head}\PY{p}{(}\PY{l+m+mi}{2}\PY{p}{)}
\end{Verbatim}


\begin{Verbatim}[commandchars=\\\{\}]
{\color{outcolor}Out[{\color{outcolor}56}]:}   user\_id\textbackslash{}ttimestamp\textbackslash{}tgroup\textbackslash{}tlanding\_page\textbackslash{}tconverted
         0  851104\textbackslash{}t2017-01-21 22:11:48.556739\textbackslash{}tcontrol\textbackslash{}to{\ldots}
         1  804228\textbackslash{}t2017-01-12 08:01:45.159739\textbackslash{}tcontrol\textbackslash{}to{\ldots}
\end{Verbatim}
            
    \begin{enumerate}
\def\labelenumi{\roman{enumi}.}
\tightlist
\item
  Plot a histogram of the \textbf{p\_diffs}. Does this plot look like
  what you expected? Use the matching problem in the classroom to assure
  you fully understand what was computed here.
\end{enumerate}

    \begin{Verbatim}[commandchars=\\\{\}]
{\color{incolor}In [{\color{incolor}27}]:} \PY{o}{\PYZpc{}}\PY{k}{matplotlib} inline
         \PY{k+kn}{import} \PY{n+nn}{seaborn} \PY{k}{as} \PY{n+nn}{sns}
         \PY{k+kn}{import} \PY{n+nn}{matplotlib}\PY{n+nn}{.}\PY{n+nn}{pyplot} \PY{k}{as} \PY{n+nn}{plt}
         
         \PY{n}{sns}\PY{o}{.}\PY{n}{set\PYZus{}style}\PY{p}{(}\PY{l+s+s2}{\PYZdq{}}\PY{l+s+s2}{darkgrid}\PY{l+s+s2}{\PYZdq{}}\PY{p}{)} 
         \PY{n}{sns}\PY{o}{.}\PY{n}{distplot}\PY{p}{(}\PY{n}{p\PYZus{}diffs}\PY{p}{)}\PY{o}{.}\PY{n}{set\PYZus{}title}\PY{p}{(}\PY{l+s+s1}{\PYZsq{}}\PY{l+s+s1}{Histograma das p\PYZus{}diffs}\PY{l+s+s1}{\PYZsq{}}\PY{p}{)}\PY{p}{;} \PY{c+c1}{\PYZsh{}\PYZlt{}\PYZhy{}esse ; elimina aquela linha de chamada horrorosa}
         
         \PY{c+c1}{\PYZsh{}código descartado (eu posso precisar dele, se algo der errado!)}
         \PY{c+c1}{\PYZsh{}plt.ioff()}
         \PY{c+c1}{\PYZsh{}seaborn.distplot(a, bins=None, hist=True, kde=True, rug=False, fit=None, }
         \PY{c+c1}{\PYZsh{}                 hist\PYZus{}kws=None, kde\PYZus{}kws=None, rug\PYZus{}kws=None, fit\PYZus{}kws=None}
         \PY{c+c1}{\PYZsh{}                 color=None, vertical=False, norm\PYZus{}hist=False, axlabel=None, label=None, ax=None)}
         
         \PY{c+c1}{\PYZsh{}plt.hist(p\PYZus{}diffs) \PYZsh{}ficou horroroso, vamos para o Seaborn!}
\end{Verbatim}


    \begin{center}
    \adjustimage{max size={0.9\linewidth}{0.9\paperheight}}{output_96_0.png}
    \end{center}
    { \hspace*{\fill} \\}
    
    Queremos a diferença da média de conversões ocorridas na página nova e
na antiga:

    \begin{Verbatim}[commandchars=\\\{\}]
{\color{incolor}In [{\color{incolor}77}]:} \PY{c+c1}{\PYZsh{}fizemos essas coisas lá em cima...}
         \PY{n}{dfnovo} \PY{o}{=} \PY{n}{df2}\PY{o}{.}\PY{n}{query}\PY{p}{(}\PY{l+s+s1}{\PYZsq{}}\PY{l+s+s1}{group == }\PY{l+s+s1}{\PYZdq{}}\PY{l+s+s1}{treatment}\PY{l+s+s1}{\PYZdq{}}\PY{l+s+s1}{\PYZsq{}}\PY{p}{)}
         \PY{n}{medconvnovo} \PY{o}{=} \PY{n}{dfnovo}\PY{o}{.}\PY{n}{converted}\PY{o}{.}\PY{n}{mean}\PY{p}{(}\PY{p}{)}
         
         \PY{n}{dfvelho} \PY{o}{=} \PY{n}{df2}\PY{o}{.}\PY{n}{query}\PY{p}{(}\PY{l+s+s1}{\PYZsq{}}\PY{l+s+s1}{group == }\PY{l+s+s1}{\PYZdq{}}\PY{l+s+s1}{control}\PY{l+s+s1}{\PYZdq{}}\PY{l+s+s1}{\PYZsq{}}\PY{p}{)}
         \PY{n}{medconvvelho} \PY{o}{=} \PY{n}{dfvelho}\PY{o}{.}\PY{n}{converted}\PY{o}{.}\PY{n}{mean}\PY{p}{(}\PY{p}{)}
         
         \PY{n}{diffobservada} \PY{o}{=} \PY{n}{medconvnovo} \PY{o}{\PYZhy{}} \PY{n}{medconvvelho}
         \PY{n+nb}{print}\PY{p}{(}\PY{n}{diffobservada}\PY{p}{)}
         
         \PY{c+c1}{\PYZsh{}plt.axvline(x=obs\PYZus{}diff, color=\PYZsq{}red\PYZsq{});}
         \PY{n}{plt}\PY{o}{.}\PY{n}{axvline}\PY{p}{(}\PY{n}{x}\PY{o}{=}\PY{n}{diffobservada}\PY{p}{,} \PY{n}{linestyle} \PY{o}{=} \PY{l+s+s1}{\PYZsq{}}\PY{l+s+s1}{\PYZhy{}\PYZhy{}}\PY{l+s+s1}{\PYZsq{}}\PY{p}{,} \PY{n}{color}\PY{o}{=}\PY{l+s+s1}{\PYZsq{}}\PY{l+s+s1}{g}\PY{l+s+s1}{\PYZsq{}}\PY{p}{)}\PY{p}{;}
\end{Verbatim}


    \begin{Verbatim}[commandchars=\\\{\}]
-0.0015790565976871451

    \end{Verbatim}

    \begin{center}
    \adjustimage{max size={0.9\linewidth}{0.9\paperheight}}{output_98_1.png}
    \end{center}
    { \hspace*{\fill} \\}
    
    Juntando as coisas\ldots{}

\emph{Bom, eu espeava essa linha verde aí bem no \textbf{centro} da
minha curva de distribuição normal\ldots{}}

\emph{Pensando bem, o azulzinho veio de uma simulação computacional, o
verdinho veio de um outro dataset, que é o do experimento real\ldots{}}

\emph{Se eu não estivesse com meu gerador de números aleatórios travado
em um set de geração, eu aposto como se se fosse rodando esta simulação
várias vezes, iria ver essa barra verde ``passeando'' pelo gráfico de
distribuição normal\ldots{} eu posso dizer que sei qual é o problema
aqui: é \textbf{ruído}! Estou lidando com um conjunto de amostras
\textbf{muito grande} e como dito em aula, o ruído pode nos pregar
alguns truques!}

\begin{center}\rule{0.5\linewidth}{\linethickness}\end{center}

\emph{A propósito, ando penando mais do que esperava para gerar meus
gráficos\ldots{} então estou adquirindo o tutorial da Datacamp de
plotagens no Seaborn (o Pyplot sempre me deu surpresas horríveis!)}

    \begin{Verbatim}[commandchars=\\\{\}]
{\color{incolor}In [{\color{incolor}72}]:} \PY{n}{bsns}\PY{o}{.}\PY{n}{set\PYZus{}style}\PY{p}{(}\PY{l+s+s2}{\PYZdq{}}\PY{l+s+s2}{darkgrid}\PY{l+s+s2}{\PYZdq{}}\PY{p}{)} 
         \PY{n}{sns}\PY{o}{.}\PY{n}{distplot}\PY{p}{(}\PY{n}{p\PYZus{}diffs}\PY{p}{)}\PY{o}{.}\PY{n}{set\PYZus{}title}\PY{p}{(}\PY{l+s+s1}{\PYZsq{}}\PY{l+s+s1}{Histograma das p\PYZus{}diffs}\PY{l+s+s1}{\PYZsq{}}\PY{p}{)}
         \PY{n}{plt}\PY{o}{.}\PY{n}{axvline}\PY{p}{(}\PY{n}{x}\PY{o}{=}\PY{n}{diffobservada}\PY{p}{,} \PY{n}{linestyle} \PY{o}{=} \PY{l+s+s1}{\PYZsq{}}\PY{l+s+s1}{\PYZhy{}\PYZhy{}}\PY{l+s+s1}{\PYZsq{}}\PY{p}{,} \PY{n}{color}\PY{o}{=}\PY{l+s+s1}{\PYZsq{}}\PY{l+s+s1}{g}\PY{l+s+s1}{\PYZsq{}}\PY{p}{)}\PY{p}{;}
\end{Verbatim}


    \begin{center}
    \adjustimage{max size={0.9\linewidth}{0.9\paperheight}}{output_100_0.png}
    \end{center}
    { \hspace*{\fill} \\}
    
    \begin{enumerate}
\def\labelenumi{\alph{enumi}.}
\setcounter{enumi}{9}
\tightlist
\item
  What proportion of the \textbf{p\_diffs} are greater than the actual
  difference observed in \textbf{ab\_data.csv}?
\end{enumerate}

    Isso daqui tem cara de \textbf{valor-p}\ldots{}

    \begin{Verbatim}[commandchars=\\\{\}]
{\color{incolor}In [{\color{incolor}75}]:} \PY{p}{(}\PY{n}{p\PYZus{}diffs} \PY{o}{\PYZgt{}} \PY{n}{diffobservada}\PY{p}{)}\PY{o}{.}\PY{n}{mean}\PY{p}{(}\PY{p}{)}
\end{Verbatim}


\begin{Verbatim}[commandchars=\\\{\}]
{\color{outcolor}Out[{\color{outcolor}75}]:} 0.9106
\end{Verbatim}
            
    \begin{enumerate}
\def\labelenumi{\alph{enumi}.}
\setcounter{enumi}{10}
\tightlist
\item
  In words, explain what you just computed in part \textbf{j.} What is
  this value called in scientific studies? What does this value mean in
  terms of whether or not there is a difference between the new and old
  pages?
\end{enumerate}

    Primeiro, com relação a \textbf{Teste de hipóteses estatísticas}, existe
o conceito de:

\begin{itemize}
\item
  \textbf{Nível de significância} que é a probabilidade de se observar
  resultados amostrais tão ou mais extremos do que aqueles observados,
  considerando-se \(H_{0}\) verdadeira (nossa \textbf{premissa
  inicial}). Se esta probabilidade for pequena, concluímos que há provas
  suficientes para rejeitar \(H_{0}\). Para isso a abordagem:
\item
  \textbf{Nível de significância observado - }valor-P**

  \begin{itemize}
  \item
    \textbf{primeiro} a estatística teste é calculada usando-se dado
    amostral
  \item
    \textbf{então} a distribuição de probabilidade apropriada é usada
    para encontrar a probabilide de se observar uma estatística amostral
    que seja pelo menos \textbf{um pouco} diferente do valor da
    \(H_{0}\) para o parâmetro populacional (\textbf{valor-P})
  \item
    (lembrando que quanto **menor o valor-p, melhor a prova contra
    \(H_{0}\)
  \end{itemize}
\end{itemize}

(fonte: \textbf{Resumão de Estatística} (texto ligeiramente modificado)
- nada como uma visita na Livraria Cultura no feriadão do
Carnaval\ldots{})

Bom, o valor-p foi calculado, apontando a proporção dos valores da
distribuição nula que foram maiores do que a diferença observada nos
dados reais.

Meu valor-p ficou \textbf{bem acima} da nossa taxa de aceitação de 5\%
esperado para erro do Tipo I (\textbf{falso positivo}, ou
\textbf{rejeitar} a \$H\_\{0\}, sendo ela \textbf{verdadeira}).

Assim podemos afirmar que:

\begin{itemize}
\item
  falhamos em rejeitar a \$H\_\{0\}
\item
  não há evidências que comprovem que a nova página produz mais
  conversões do que a primeira
\end{itemize}

    \begin{enumerate}
\def\labelenumi{\alph{enumi}.}
\setcounter{enumi}{11}
\tightlist
\item
  We could also use a built-in to achieve similar results. Though using
  the built-in might be easier to code, the above portions are a
  walkthrough of the ideas that are critical to correctly thinking about
  statistical significance. Fill in the below to calculate the number of
  conversions for each page, as well as the number of individuals who
  received each page. Let \texttt{n\_old} and \texttt{n\_new} refer the
  the number of rows associated with the old page and new pages,
  respectively.
\end{enumerate}

    \begin{Verbatim}[commandchars=\\\{\}]
{\color{incolor}In [{\color{incolor}24}]:} \PY{n}{dfnovo2} \PY{o}{=} \PY{n}{df2}\PY{o}{.}\PY{n}{query}\PY{p}{(}\PY{l+s+s1}{\PYZsq{}}\PY{l+s+s1}{landing\PYZus{}page == }\PY{l+s+s1}{\PYZdq{}}\PY{l+s+s1}{new\PYZus{}page}\PY{l+s+s1}{\PYZdq{}}\PY{l+s+s1}{\PYZsq{}}\PY{p}{)}
         \PY{n}{medconvnovo} \PY{o}{=} \PY{n}{dfnovo2}\PY{o}{.}\PY{n}{converted}\PY{o}{.}\PY{n}{mean}\PY{p}{(}\PY{p}{)}
         
         \PY{n}{dfvelho2} \PY{o}{=} \PY{n}{df2}\PY{o}{.}\PY{n}{query}\PY{p}{(}\PY{l+s+s1}{\PYZsq{}}\PY{l+s+s1}{landing\PYZus{}page == }\PY{l+s+s1}{\PYZdq{}}\PY{l+s+s1}{old\PYZus{}page}\PY{l+s+s1}{\PYZdq{}}\PY{l+s+s1}{\PYZsq{}}\PY{p}{)}
         \PY{n}{medconvvelho} \PY{o}{=} \PY{n}{dfvelho2}\PY{o}{.}\PY{n}{converted}\PY{o}{.}\PY{n}{mean}\PY{p}{(}\PY{p}{)}
         
         \PY{n}{convert\PYZus{}new} \PY{o}{=} \PY{n}{dfnovo2}\PY{p}{[}\PY{l+s+s1}{\PYZsq{}}\PY{l+s+s1}{converted}\PY{l+s+s1}{\PYZsq{}}\PY{p}{]}\PY{o}{.}\PY{n}{sum}\PY{p}{(}\PY{p}{)}
         \PY{n}{convert\PYZus{}old} \PY{o}{=} \PY{n}{dfvelho2}\PY{p}{[}\PY{l+s+s1}{\PYZsq{}}\PY{l+s+s1}{converted}\PY{l+s+s1}{\PYZsq{}}\PY{p}{]}\PY{o}{.}\PY{n}{sum}\PY{p}{(}\PY{p}{)}
         \PY{n+nb}{print} \PY{p}{(}\PY{l+s+s1}{\PYZsq{}}\PY{l+s+s1}{Convertidos \PYZhy{} antiga :}\PY{l+s+si}{\PYZob{}\PYZcb{}}\PY{l+s+s1}{ nova :}\PY{l+s+si}{\PYZob{}\PYZcb{}}\PY{l+s+s1}{\PYZsq{}}\PY{o}{.}\PY{n}{format}\PY{p}{(}\PY{n}{convert\PYZus{}old}\PY{p}{,} \PY{n}{convert\PYZus{}new}\PY{p}{)}\PY{p}{)}
         
         \PY{n}{n\PYZus{}new} \PY{o}{=} \PY{n+nb}{len}\PY{p}{(}\PY{n}{dfnovo2}\PY{p}{)}
         \PY{n}{n\PYZus{}old} \PY{o}{=} \PY{n+nb}{len}\PY{p}{(}\PY{n}{dfvelho2}\PY{p}{)}
         \PY{n+nb}{print} \PY{p}{(}\PY{l+s+s1}{\PYZsq{}}\PY{l+s+s1}{Acessos \PYZhy{} antiga :}\PY{l+s+si}{\PYZob{}\PYZcb{}}\PY{l+s+s1}{ nova :}\PY{l+s+si}{\PYZob{}\PYZcb{}}\PY{l+s+s1}{\PYZsq{}}\PY{o}{.}\PY{n}{format}\PY{p}{(}\PY{n}{n\PYZus{}old}\PY{p}{,} \PY{n}{n\PYZus{}new}\PY{p}{)}\PY{p}{)}
\end{Verbatim}


    \begin{Verbatim}[commandchars=\\\{\}]
Convertidos - antiga :17489 nova :17264
Acessos - antiga :145274 nova :145310

    \end{Verbatim}

    \begin{enumerate}
\def\labelenumi{\alph{enumi}.}
\setcounter{enumi}{12}
\tightlist
\item
  Now use \texttt{stats.proportions\_ztest} to compute your test
  statistic and p-value.
  \href{http://knowledgetack.com/python/statsmodels/proportions_ztest/}{Here}
  is a helpful link on using the built in.
\end{enumerate}

    O link indicado não funciona, o site não existe mais! Mas no
Statsmodels, encontrei
\href{https://www.statsmodels.org/dev/generated/statsmodels.stats.proportion.proportions_ztest.html}{aqui}
algumas dicas bastante claras sobre o uso do método
\textbf{proportions\_ztest()}

\begin{center}\rule{0.5\linewidth}{\linethickness}\end{center}

\begin{verbatim}
    .stats.proportions_ztest(count, nobs, value=None, alternative='two-sided', prop_var=False)
\end{verbatim}

\begin{itemize}
\item
  \textbf{count} -- the number of successes in nobs trials. If this is
  array\_like, then the assumption is that this represents the number of
  successes for each independent sample
\item
  \textbf{nobs} -- the number of trials or observations, with the same
  length as count
\end{itemize}

\begin{center}\rule{0.5\linewidth}{\linethickness}\end{center}

\begin{itemize}
\item
  \textbf{alternative} -- The alternative hypothesis can be either
  two-sided (\textbf{não é meu caso}) or one of the one-sided tests
\item
  In the two sample test:
\item
  smaller means that the alternative hypothesis is \textbf{prop
  \textless{} value} (p1 \textless{} p2) (\textbf{não quero isso})
\item
  larger means \textbf{prop \textgreater{} value} (p1 \textgreater{} p2
  - where p1 is the proportion of the first sample and p2 of the second
  one)
\end{itemize}

\begin{center}\rule{0.5\linewidth}{\linethickness}\end{center}

Returns:

\begin{itemize}
\tightlist
\item
  \textbf{zstat} -- test statistic for the z-test
\item
  \textbf{p-value} -- p-value for the z-test
\end{itemize}

\emph{Fonte: Statsmodels (levemente adaptado para meu contexto)}

    \begin{Verbatim}[commandchars=\\\{\}]
{\color{incolor}In [{\color{incolor}40}]:} \PY{k+kn}{import} \PY{n+nn}{statsmodels}\PY{n+nn}{.}\PY{n+nn}{api} \PY{k}{as} \PY{n+nn}{sm}
\end{Verbatim}


    \begin{Verbatim}[commandchars=\\\{\}]
{\color{incolor}In [{\color{incolor}41}]:} \PY{k+kn}{from} \PY{n+nn}{scipy}\PY{n+nn}{.}\PY{n+nn}{stats} \PY{k}{import} \PY{n}{norm}
\end{Verbatim}


    \begin{Verbatim}[commandchars=\\\{\}]
{\color{incolor}In [{\color{incolor}28}]:} \PY{c+c1}{\PYZsh{}sm.stats.proportion.proportions\PYZus{}ztest(count, nobs, }
         \PY{c+c1}{\PYZsh{}                                      value=None, alternative=\PYZsq{}two\PYZhy{}sided\PYZsq{}, prop\PYZus{}var=False}
         \PY{n}{valores} \PY{o}{=} \PY{n}{sm}\PY{o}{.}\PY{n}{stats}\PY{o}{.}\PY{n}{proportions\PYZus{}ztest}\PY{p}{(}\PY{p}{[}\PY{n}{convert\PYZus{}new}\PY{p}{,} \PY{n}{convert\PYZus{}old}\PY{p}{]}\PY{p}{,} \PY{p}{[}\PY{n}{n\PYZus{}new}\PY{p}{,} \PY{n}{n\PYZus{}old}\PY{p}{]}\PY{p}{,} 
                                              \PY{n}{alternative}\PY{o}{=}\PY{l+s+s1}{\PYZsq{}}\PY{l+s+s1}{larger}\PY{l+s+s1}{\PYZsq{}}\PY{p}{)}
         \PY{c+c1}{\PYZsh{}isso me traz uma tupla (teste\PYZhy{}z, valor\PYZhy{}p)}
         \PY{n+nb}{print}\PY{p}{(}\PY{l+s+s1}{\PYZsq{}}\PY{l+s+s1}{teste z (unicaudal): }\PY{l+s+s1}{\PYZsq{}}\PY{p}{,} \PY{n}{valores}\PY{p}{[}\PY{l+m+mi}{0}\PY{p}{]}\PY{p}{)}
         \PY{n+nb}{print}\PY{p}{(}\PY{l+s+s1}{\PYZsq{}}\PY{l+s+s1}{valor\PYZhy{}p (unicaudal): }\PY{l+s+s1}{\PYZsq{}}\PY{p}{,} \PY{n}{valores}\PY{p}{[}\PY{l+m+mi}{1}\PY{p}{]}\PY{p}{)}
\end{Verbatim}


    \begin{Verbatim}[commandchars=\\\{\}]
teste z (unicaudal):  -1.3109241984234394
valor-p (unicaudal):  0.9050583127590245

    \end{Verbatim}

    \begin{center}\rule{0.5\linewidth}{\linethickness}\end{center}

Isso tem a ver com \textbf{Estatística-T}

Dada uma curva de distribuição \textbf{normal}, nós podemos inferir
áreas sob esta curva, cruzadas por pontos de probabilizade \(z\).

*Antigamente tínhamos tabelas para fazer isso. Então eu peguei o meu
\textbf{Resumão de Estatística}, coloquei lá \textbf{Teste Unicaudal},
com \textbf{Nível de Significância A = 0.05} (este A me diz que é um
teste \textbf{unilateral}, ou \textbf{unicaudal}). Muito bem, eu quero a
rabeira da curva, com \textbf{GL = inf} (graus de liberdade infinitos,
ou maior do que 30) e chego a:

\begin{itemize}
\tightlist
\item
  \(\alpha = 1.645\)
\end{itemize}

\hypertarget{como-calcular-valores-cruxedticos}{%
\paragraph{Como calcular valores
críticos}\label{como-calcular-valores-cruxedticos}}

\textbf{Cumulative Density Function (CDF)}: Returns the probability for
an observation equal to or lesser than a specific value from the
distribution.

\textbf{Percent Point Function (PPF)}: Returns the observation value for
the provided probability that is less than or equal to the provided
probability from the distribution (o contrário do CDF - é o que eu
procuro!)

\emph{Conceitos encontrados
\href{https://machinelearningmastery.com/critical-values-for-statistical-hypothesis-testing/}{aqui}}

\begin{center}\rule{0.5\linewidth}{\linethickness}\end{center}

    \begin{Verbatim}[commandchars=\\\{\}]
{\color{incolor}In [{\color{incolor}101}]:} \PY{n}{estatistica} \PY{o}{=} \PY{l+m+mf}{0.95}
          \PY{n}{alpha} \PY{o}{=} \PY{n}{norm}\PY{o}{.}\PY{n}{ppf}\PY{p}{(}\PY{n}{estatistica}\PY{p}{)}
          \PY{n+nb}{print}\PY{p}{(}\PY{l+s+s1}{\PYZsq{}}\PY{l+s+s1}{alfa: }\PY{l+s+s1}{\PYZsq{}}\PY{p}{,} \PY{n}{alpha}\PY{p}{)}
\end{Verbatim}


    \begin{Verbatim}[commandchars=\\\{\}]
alfa:  1.6448536269514722

    \end{Verbatim}

    \begin{enumerate}
\def\labelenumi{\alph{enumi}.}
\setcounter{enumi}{13}
\tightlist
\item
  What do the z-score and p-value you computed in the previous question
  mean for the conversion rates of the old and new pages? Do they agree
  with the findings in parts \textbf{j.} and \textbf{k.}?
\end{enumerate}

    Eu tenho:

\begin{itemize}
\item
  um teste \textbf{unicaudal}
\item
  um \(\alpha=1.645\) (meu valor crítico para este teste)
\end{itemize}

Eu tive comos resultados:

\begin{itemize}
\item
  Escore-Z de \textbf{-1.31} (abaixo do meu \(\alpha\) - e localizado
  \textbf{antes} do meu ponto central da distribuição normal)
\item
  Valor-P de \textbf{0.91} (acima de 0.05, ou 5\%)
\end{itemize}

Com estes resultados, novamente \textbf{falhamos} em rejeitar \(H_{0}\)

Em outras palavras, \(H_{0}\) continua \textbf{válida}, ou seja, que a
página \textbf{anterior} era \textbf{tão boa} (ou até melhor) do que a
página \textbf{nova}, em produzir conversões.

    \hypertarget{parei-aqui}{%
\paragraph{Parei aqui}\label{parei-aqui}}

    \hypertarget{section}{%
\subsubsection{5/5}\label{section}}

 \#\#\# Part III - A regression approach

\texttt{1.} In this final part, you will see that the result you
acheived in the previous A/B test can also be acheived by performing
regression.

\begin{enumerate}
\def\labelenumi{\alph{enumi}.}
\tightlist
\item
  Since each row is either a conversion or no conversion, what type of
  regression should you be performing in this case?
\end{enumerate}

    Segundo o curso, a regressão que melhor se adapta à conversão de {[}0,
1{]}, tipo sim/não é a regressão \textbf{logística}. O que faz sentido,
pois ela é altamente \textbf{dispersiva}, ou seja, qualquer afastamento
do seu núcleo central leva a uma decisão. No caso, o sim/não pode ser
interpretado como uma variável \textbf{categórica}

Uma explicação melhor para variáveis dummie se encontra {[}aqui{]}.
Historicamente, máquinas de \textbf{babbage} eram programadas com fitas
perfuradas (na verdade eram teares automatizados). Quando você tem uma
categoria com n tipos diferentes, matematicamente a melhor forma é
tratar cada tipo como sendo uma \textbf{feição} diferente. Assim,
testa-se uma a uma quais as alterações que ela causa no nosso modelo.
Faz sentido.

    \begin{enumerate}
\def\labelenumi{\alph{enumi}.}
\setcounter{enumi}{1}
\tightlist
\item
  The goal is to use \textbf{statsmodels} to fit the regression model
  you specified in part \textbf{a.} to see if there is a significant
  difference in conversion based on which page a customer receives.
\end{enumerate}

However, you first need to

\begin{itemize}
\item
  create a column for the intercept, and
\item
  create a dummy variable column for which page each user received.
\end{itemize}

Add an \textbf{intercept} column, as well as an \textbf{ab\_page}
column, which is

\begin{itemize}
\item
  1 when an individual receives the \textbf{treatment} and
\item
  0 if \textbf{control}.
\end{itemize}

    Eu preciso criar a variável dummie, com 0 para \textbf{control} e 1 para
\textbf{treatment}

    \begin{Verbatim}[commandchars=\\\{\}]
{\color{incolor}In [{\color{incolor}34}]:} \PY{c+c1}{\PYZsh{}df[[\PYZsq{}no\PYZus{}fraud\PYZsq{}, \PYZsq{}fraud\PYZsq{}]] = pd.get\PYZus{}dummies(df[\PYZsq{}fraud\PYZsq{}])}
         \PY{c+c1}{\PYZsh{}df = df.drop(\PYZsq{}no\PYZus{}fraud\PYZsq{}, axis = 1)}
         
         \PY{n}{df2}\PY{p}{[}\PY{p}{[}\PY{l+s+s1}{\PYZsq{}}\PY{l+s+s1}{control}\PY{l+s+s1}{\PYZsq{}}\PY{p}{,} \PY{l+s+s1}{\PYZsq{}}\PY{l+s+s1}{treatment}\PY{l+s+s1}{\PYZsq{}}\PY{p}{]}\PY{p}{]} \PY{o}{=} \PY{n}{pd}\PY{o}{.}\PY{n}{get\PYZus{}dummies}\PY{p}{(}\PY{n}{df2}\PY{p}{[}\PY{l+s+s1}{\PYZsq{}}\PY{l+s+s1}{group}\PY{l+s+s1}{\PYZsq{}}\PY{p}{]}\PY{p}{)}
         \PY{n}{df2}\PY{o}{.}\PY{n}{head}\PY{p}{(}\PY{l+m+mi}{4}\PY{p}{)}
\end{Verbatim}


\begin{Verbatim}[commandchars=\\\{\}]
{\color{outcolor}Out[{\color{outcolor}34}]:}                           timestamp      group landing\_page  converted  \textbackslash{}
         user\_id                                                                  
         851104   2017-01-21 22:11:48.556739    control     old\_page          0   
         804228   2017-01-12 08:01:45.159739    control     old\_page          0   
         661590   2017-01-11 16:55:06.154213  treatment     new\_page          0   
         853541   2017-01-08 18:28:03.143765  treatment     new\_page          0   
         
                  control  treatment  
         user\_id                      
         851104         1          0  
         804228         1          0  
         661590         0          1  
         853541         0          1  
\end{Verbatim}
            
    Para não dar sobrevinculação, eu posso remover uma das minhas dummies:

\begin{itemize}
\tightlist
\item
  no caso, \textbf{control}, pois \textbf{treatment} se alinha para o
  que queremos na nossa \textbf{ab\_page}
\end{itemize}

    \begin{Verbatim}[commandchars=\\\{\}]
{\color{incolor}In [{\color{incolor}35}]:} \PY{n}{df2} \PY{o}{=} \PY{n}{df2}\PY{o}{.}\PY{n}{drop}\PY{p}{(}\PY{l+s+s1}{\PYZsq{}}\PY{l+s+s1}{control}\PY{l+s+s1}{\PYZsq{}}\PY{p}{,} \PY{n}{axis} \PY{o}{=} \PY{l+m+mi}{1}\PY{p}{)}
         \PY{n}{df2}\PY{o}{.}\PY{n}{head}\PY{p}{(}\PY{l+m+mi}{4}\PY{p}{)}
\end{Verbatim}


\begin{Verbatim}[commandchars=\\\{\}]
{\color{outcolor}Out[{\color{outcolor}35}]:}                           timestamp      group landing\_page  converted  \textbackslash{}
         user\_id                                                                  
         851104   2017-01-21 22:11:48.556739    control     old\_page          0   
         804228   2017-01-12 08:01:45.159739    control     old\_page          0   
         661590   2017-01-11 16:55:06.154213  treatment     new\_page          0   
         853541   2017-01-08 18:28:03.143765  treatment     new\_page          0   
         
                  treatment  
         user\_id             
         851104           0  
         804228           0  
         661590           1  
         853541           1  
\end{Verbatim}
            
    \begin{Verbatim}[commandchars=\\\{\}]
{\color{incolor}In [{\color{incolor}42}]:} \PY{n}{df2}\PY{o}{.}\PY{n}{columns} \PY{o}{=} \PY{p}{[}\PY{l+s+s1}{\PYZsq{}}\PY{l+s+s1}{user\PYZus{}id}\PY{l+s+s1}{\PYZsq{}}\PY{p}{,} \PY{l+s+s1}{\PYZsq{}}\PY{l+s+s1}{timestamp}\PY{l+s+s1}{\PYZsq{}}\PY{p}{,} \PY{l+s+s1}{\PYZsq{}}\PY{l+s+s1}{group}\PY{l+s+s1}{\PYZsq{}}\PY{p}{,} \PY{l+s+s1}{\PYZsq{}}\PY{l+s+s1}{converted}\PY{l+s+s1}{\PYZsq{}}\PY{p}{,} \PY{l+s+s1}{\PYZsq{}}\PY{l+s+s1}{ab\PYZus{}page}\PY{l+s+s1}{\PYZsq{}}\PY{p}{]}
         \PY{n}{df2}\PY{o}{.}\PY{n}{head}\PY{p}{(}\PY{l+m+mi}{4}\PY{p}{)}
\end{Verbatim}


\begin{Verbatim}[commandchars=\\\{\}]
{\color{outcolor}Out[{\color{outcolor}42}]:}                             user\_id  timestamp     group  converted  ab\_page
         user\_id                                                                     
         851104   2017-01-21 22:11:48.556739    control  old\_page          0        0
         804228   2017-01-12 08:01:45.159739    control  old\_page          0        0
         661590   2017-01-11 16:55:06.154213  treatment  new\_page          0        1
         853541   2017-01-08 18:28:03.143765  treatment  new\_page          0        1
\end{Verbatim}
            
    Adicionar o intercept:

    \begin{Verbatim}[commandchars=\\\{\}]
{\color{incolor}In [{\color{incolor}43}]:} \PY{n}{df2}\PY{p}{[}\PY{l+s+s1}{\PYZsq{}}\PY{l+s+s1}{intercept}\PY{l+s+s1}{\PYZsq{}}\PY{p}{]} \PY{o}{=} \PY{l+m+mi}{1}
         \PY{n}{df2}\PY{o}{.}\PY{n}{head}\PY{p}{(}\PY{l+m+mi}{4}\PY{p}{)}
\end{Verbatim}


\begin{Verbatim}[commandchars=\\\{\}]
{\color{outcolor}Out[{\color{outcolor}43}]:}                             user\_id  timestamp     group  converted  ab\_page  \textbackslash{}
         user\_id                                                                        
         851104   2017-01-21 22:11:48.556739    control  old\_page          0        0   
         804228   2017-01-12 08:01:45.159739    control  old\_page          0        0   
         661590   2017-01-11 16:55:06.154213  treatment  new\_page          0        1   
         853541   2017-01-08 18:28:03.143765  treatment  new\_page          0        1   
         
                  intercept  
         user\_id             
         851104           1  
         804228           1  
         661590           1  
         853541           1  
\end{Verbatim}
            
    Não use esta linha: (apenas para corrigir erros!)

    \begin{Verbatim}[commandchars=\\\{\}]
{\color{incolor}In [{\color{incolor}41}]:} \PY{n}{df2} \PY{o}{=} \PY{n}{df2}\PY{o}{.}\PY{n}{drop}\PY{p}{(}\PY{l+s+s1}{\PYZsq{}}\PY{l+s+s1}{intercept}\PY{l+s+s1}{\PYZsq{}}\PY{p}{,} \PY{n}{axis} \PY{o}{=} \PY{l+m+mi}{1}\PY{p}{)}
\end{Verbatim}


    \begin{enumerate}
\def\labelenumi{\alph{enumi}.}
\setcounter{enumi}{2}
\tightlist
\item
  Use \textbf{statsmodels} to import your regression model.
\end{enumerate}

\begin{itemize}
\item
  Instantiate the model, and
\item
  fit the model using the two columns you created in part \textbf{b.} to
  predict whether or not an individual converts.
\end{itemize}

    Instanciando:

    \begin{Verbatim}[commandchars=\\\{\}]
{\color{incolor}In [{\color{incolor}45}]:} \PY{c+c1}{\PYZsh{}sm.Logit(df[\PYZsq{}fraud\PYZsq{}], axis=1) \PYZsh{}o primeiro argumento é a resposta}
         \PY{c+c1}{\PYZsh{}logit\PYZus{}mod = sm.Logit(df[\PYZsq{}fraud\PYZsq{}], df[[\PYZsq{}intercept\PYZsq{}, \PYZsq{}duration\PYZsq{}]])}
         
         \PY{c+c1}{\PYZsh{}sm.Logit(df2[\PYZsq{}converted\PYZsq{}], axis=1)}
         \PY{n}{modlog} \PY{o}{=} \PY{n}{sm}\PY{o}{.}\PY{n}{Logit}\PY{p}{(}\PY{n}{df2}\PY{p}{[}\PY{l+s+s1}{\PYZsq{}}\PY{l+s+s1}{converted}\PY{l+s+s1}{\PYZsq{}}\PY{p}{]}\PY{p}{,} \PY{n}{df2}\PY{p}{[}\PY{p}{[}\PY{l+s+s1}{\PYZsq{}}\PY{l+s+s1}{intercept}\PY{l+s+s1}{\PYZsq{}}\PY{p}{,} \PY{l+s+s1}{\PYZsq{}}\PY{l+s+s1}{ab\PYZus{}page}\PY{l+s+s1}{\PYZsq{}}\PY{p}{]}\PY{p}{]}\PY{p}{)}
\end{Verbatim}


    Ajustando:

    \begin{Verbatim}[commandchars=\\\{\}]
{\color{incolor}In [{\color{incolor}46}]:} \PY{n}{results} \PY{o}{=} \PY{n}{modlog}\PY{o}{.}\PY{n}{fit}\PY{p}{(}\PY{p}{)}
         \PY{n}{results}\PY{o}{.}\PY{n}{summary}
\end{Verbatim}


    \begin{Verbatim}[commandchars=\\\{\}]
Optimization terminated successfully.
         Current function value: 0.366118
         Iterations 6

    \end{Verbatim}

\begin{Verbatim}[commandchars=\\\{\}]
{\color{outcolor}Out[{\color{outcolor}46}]:} <bound method BinaryResults.summary of <statsmodels.discrete.discrete\_model.LogitResults object at 0x0000021B8511F518>>
\end{Verbatim}
            
    Lembretes da aula de logística: (não devo rodar estas linhas)

    \begin{Verbatim}[commandchars=\\\{\}]
{\color{incolor}In [{\color{incolor} }]:} \PY{c+c1}{\PYZsh{} eu preciso exponenciar o coeficiente da duração}
        \PY{n}{np}\PY{o}{.}\PY{n}{exp}\PY{p}{(}\PY{o}{\PYZhy{}}\PY{l+m+mf}{1.4637}\PY{p}{)} \PY{c+c1}{\PYZsh{}para cada unidade aumentada na duração, fraude é 0.23 mais provável}
        
        \PY{c+c1}{\PYZsh{}o recíproco}
        \PY{l+m+mi}{1}\PY{o}{/}\PY{n}{np}\PY{o}{.}\PY{n}{exp}\PY{p}{(}\PY{o}{\PYZhy{}}\PY{l+m+mf}{1.4637}\PY{p}{)} \PY{c+c1}{\PYZsh{}4.32 vezes menos provável para cada unidade diminuída na duração}
        
        \PY{c+c1}{\PYZsh{}e o do weekday}
        \PY{n}{np}\PY{o}{.}\PY{n}{exp}\PY{p}{(}\PY{l+m+mf}{2.5465}\PY{p}{)} \PY{c+c1}{\PYZsh{}12.75 vezes mais comum nos dias de semana do que nos finais de semana}
\end{Verbatim}


    \begin{enumerate}
\def\labelenumi{\alph{enumi}.}
\setcounter{enumi}{3}
\tightlist
\item
  Provide the summary of your model below, and use it as necessary to
  answer the following questions.
\end{enumerate}

    \begin{Verbatim}[commandchars=\\\{\}]
{\color{incolor}In [{\color{incolor}47}]:} \PY{n}{results}\PY{o}{.}\PY{n}{summary}\PY{p}{(}\PY{p}{)}
\end{Verbatim}


\begin{Verbatim}[commandchars=\\\{\}]
{\color{outcolor}Out[{\color{outcolor}47}]:} <class 'statsmodels.iolib.summary.Summary'>
         """
                                    Logit Regression Results                           
         ==============================================================================
         Dep. Variable:              converted   No. Observations:               290584
         Model:                          Logit   Df Residuals:                   290582
         Method:                           MLE   Df Model:                            1
         Date:                Thu, 07 Mar 2019   Pseudo R-squ.:               8.077e-06
         Time:                        22:06:48   Log-Likelihood:            -1.0639e+05
         converged:                       True   LL-Null:                   -1.0639e+05
                                                 LLR p-value:                    0.1899
         ==============================================================================
                          coef    std err          z      P>|z|      [0.025      0.975]
         ------------------------------------------------------------------------------
         intercept     -1.9888      0.008   -246.669      0.000      -2.005      -1.973
         ab\_page       -0.0150      0.011     -1.311      0.190      -0.037       0.007
         ==============================================================================
         """
\end{Verbatim}
            
    \begin{enumerate}
\def\labelenumi{\alph{enumi}.}
\setcounter{enumi}{4}
\tightlist
\item
  What is the p-value associated with \textbf{ab\_page}? Why does it
  differ from the value you found in \textbf{Part II}? \textbf{Hint}:
  What are the null and alternative hypotheses associated with your
  regression model, and how do they compare to the null and alternative
  hypotheses in the \textbf{Part II}?
\end{enumerate}

    Para a coluna \textbf{ab\_page}, obtive um valor-p de 0.19\ldots{} ele é
diferente do valor encontrado no exercício anterior?

Sim, mas são metodologias diferentes. Lá usávamos um \textbf{encaixe}
unicaudal. Aqui ele é bicaudal.

    Encontrei na Internet um exemplo de simulação de teste-Z bicaudal:

    \begin{Verbatim}[commandchars=\\\{\}]
{\color{incolor}In [{\color{incolor}48}]:} \PY{n}{resultados} \PY{o}{=} \PY{n}{sm}\PY{o}{.}\PY{n}{stats}\PY{o}{.}\PY{n}{proportions\PYZus{}ztest}\PY{p}{(}\PY{p}{[}\PY{n}{convert\PYZus{}new}\PY{p}{,} \PY{n}{convert\PYZus{}old}\PY{p}{]}\PY{p}{,} \PY{p}{[}\PY{n}{n\PYZus{}new}\PY{p}{,} \PY{n}{n\PYZus{}old}\PY{p}{]}\PY{p}{)}
         \PY{n}{alpha} \PY{o}{=} \PY{n}{norm}\PY{o}{.}\PY{n}{ppf}\PY{p}{(}\PY{l+m+mi}{1}\PY{o}{\PYZhy{}}\PY{p}{(}\PY{l+m+mf}{0.05}\PY{o}{/}\PY{l+m+mi}{2}\PY{p}{)}\PY{p}{)}
         \PY{n+nb}{print}\PY{p}{(}\PY{l+s+s1}{\PYZsq{}}\PY{l+s+s1}{escore\PYZhy{}z: }\PY{l+s+s1}{\PYZsq{}}\PY{p}{,} \PY{n+nb}{str}\PY{p}{(}\PY{n}{resultados}\PY{p}{[}\PY{l+m+mi}{0}\PY{p}{]}\PY{p}{)}\PY{p}{)}
         \PY{n+nb}{print}\PY{p}{(}\PY{l+s+s1}{\PYZsq{}}\PY{l+s+s1}{valor\PYZhy{}p: }\PY{l+s+s1}{\PYZsq{}}\PY{p}{,} \PY{n+nb}{str}\PY{p}{(}\PY{n}{resultados}\PY{p}{[}\PY{l+m+mi}{1}\PY{p}{]}\PY{p}{)}\PY{p}{)}
         \PY{n+nb}{print}\PY{p}{(}\PY{l+s+s1}{\PYZsq{}}\PY{l+s+s1}{alfa: }\PY{l+s+s1}{\PYZsq{}}\PY{p}{,} \PY{n+nb}{str}\PY{p}{(}\PY{n}{alpha}\PY{p}{)}\PY{p}{)}
\end{Verbatim}


    \begin{Verbatim}[commandchars=\\\{\}]
escore-z:  -1.3109241984234394
valor-p:  0.18988337448195103
alfa:  1.959963984540054

    \end{Verbatim}

    A mesma conclusão: a coluna \textbf{ab\_page} não parece ``explicar''
nossas \textbf{conversões}\ldots{}

    \begin{enumerate}
\def\labelenumi{\alph{enumi}.}
\setcounter{enumi}{5}
\tightlist
\item
  Now, you are considering other things that might influence whether or
  not an individual converts. Discuss why it is a good idea to consider
  other factors to add into your regression model. Are there any
  disadvantages to adding additional terms into your regression model?
\end{enumerate}

    Às vezes a adição de outras \textbf{feições} (perfil etário, conversões
anteriores - identificar se a pessoa já é cliente, grupos de interesse)
podem influenciar as decisões atuais, de converter ou não. Um modelo
suficientemente rico, mas não rico o suficiente para tentar ``modelizar
tudo'' parece ser o que produz os mehores resultados.

Agora imagine que eu vá adicionando novas e novas feições a meu modelo.
E coisas como a cor da camisa, ou o modelo da bicicleta, ou se a pessoa
costuma passar o feriado de final de ano na praia\ldots{} espere, isso é
\textbf{realmente} relevante?

Às vezes não. Algumas pessoas que compram aquele automóvel com ar
condicionado usam camisa branca, outros de lista\ldots{} e no nós temos
um monstro com n feições e muitas delas com \textbf{valor-p}
elevado\ldots{}

Num mundo ideal eu poderia ter montes e montes de entulho (desde que
minha casa fosse realmente grande). Mas na prática o que acontece? Essas
coisas vão me gerando \textbf{ruído} e distorcendo o modelo e logo eu
passo a recomendar que vendedores insistam em vender o \textbf{ar
condicionado}, caso o meu cliente esteja vestindo a \textbf{camisa
branca}\ldots{} o que é isso no fundo? O surgimento de um \textbf{falso
positivo} e que pode me colocar em situações realmente
constrangedoras\ldots{}

    \begin{enumerate}
\def\labelenumi{\alph{enumi}.}
\setcounter{enumi}{6}
\tightlist
\item
  Now along with testing if the conversion rate changes for different
  pages, also add an effect based on which country a user lives. You
  will need to read in the \textbf{countries.csv} dataset and merge
  together your datasets on the approporiate rows.
  \href{https://pandas.pydata.org/pandas-docs/stable/generated/pandas.DataFrame.join.html}{Here}
  are the docs for joining tables.
\end{enumerate}

Does it appear that country had an impact on conversion? Don't forget to
create dummy variables for these country columns - \textbf{Hint: You
will need two columns for the three dummy variables.} Provide the
statistical output as well as a written response to answer this
question.

    \begin{Verbatim}[commandchars=\\\{\}]
{\color{incolor}In [{\color{incolor} }]:} \PY{n}{df2} \PY{o}{=} \PY{n}{pd}\PY{o}{.}\PY{n}{read\PYZus{}csv}\PY{p}{(}\PY{l+s+s1}{\PYZsq{}}\PY{l+s+s1}{abdf2.csv}\PY{l+s+s1}{\PYZsq{}}\PY{p}{,} \PY{n}{sep}\PY{o}{=}\PY{l+s+s1}{\PYZsq{}}\PY{l+s+se}{\PYZbs{}t}\PY{l+s+s1}{\PYZsq{}}\PY{p}{,} \PY{n}{encoding}\PY{o}{=}\PY{l+s+s1}{\PYZsq{}}\PY{l+s+s1}{utf\PYZhy{}8}\PY{l+s+s1}{\PYZsq{}}\PY{p}{,} \PY{n}{index\PYZus{}col}\PY{o}{=}\PY{l+s+s1}{\PYZsq{}}\PY{l+s+s1}{user\PYZus{}id}\PY{l+s+s1}{\PYZsq{}}\PY{p}{)}
\end{Verbatim}


    \begin{Verbatim}[commandchars=\\\{\}]
{\color{incolor}In [{\color{incolor}14}]:} \PY{n}{dfpaises} \PY{o}{=} \PY{n}{pd}\PY{o}{.}\PY{n}{read\PYZus{}csv}\PY{p}{(}\PY{l+s+s1}{\PYZsq{}}\PY{l+s+s1}{./countries.csv}\PY{l+s+s1}{\PYZsq{}}\PY{p}{,} \PY{n}{sep}\PY{o}{=}\PY{l+s+s1}{\PYZsq{}}\PY{l+s+s1}{,}\PY{l+s+s1}{\PYZsq{}}\PY{p}{,} \PY{n}{encoding}\PY{o}{=}\PY{l+s+s1}{\PYZsq{}}\PY{l+s+s1}{utf\PYZhy{}8}\PY{l+s+s1}{\PYZsq{}}\PY{p}{,} \PY{n}{index\PYZus{}col}\PY{o}{=}\PY{l+s+s1}{\PYZsq{}}\PY{l+s+s1}{user\PYZus{}id}\PY{l+s+s1}{\PYZsq{}}\PY{p}{)}
\end{Verbatim}


    \begin{Verbatim}[commandchars=\\\{\}]
{\color{incolor}In [{\color{incolor}15}]:} \PY{n}{dfnovopaises} \PY{o}{=} \PY{n}{dfpaises}\PY{o}{.}\PY{n}{join}\PY{p}{(}\PY{n}{df2}\PY{p}{,} \PY{n}{how}\PY{o}{=}\PY{l+s+s1}{\PYZsq{}}\PY{l+s+s1}{inner}\PY{l+s+s1}{\PYZsq{}}\PY{p}{)}
\end{Verbatim}


    \begin{Verbatim}[commandchars=\\\{\}]
{\color{incolor}In [{\color{incolor}17}]:} \PY{n}{dfnovopaises}\PY{o}{.}\PY{n}{country}\PY{o}{.}\PY{n}{value\PYZus{}counts}\PY{p}{(}\PY{p}{)}
\end{Verbatim}


\begin{Verbatim}[commandchars=\\\{\}]
{\color{outcolor}Out[{\color{outcolor}17}]:} US    203619
         UK     72466
         CA     14499
         Name: country, dtype: int64
\end{Verbatim}
            
    \begin{Verbatim}[commandchars=\\\{\}]
{\color{incolor}In [{\color{incolor}18}]:} \PY{n}{dfnovopaises}\PY{o}{.}\PY{n}{head}\PY{p}{(}\PY{l+m+mi}{4}\PY{p}{)}
\end{Verbatim}


\begin{Verbatim}[commandchars=\\\{\}]
{\color{outcolor}Out[{\color{outcolor}18}]:}         country                   timestamp      group landing\_page  converted
         user\_id                                                                       
         834778       UK  2017-01-14 23:08:43.304998    control     old\_page          0
         928468       US  2017-01-23 14:44:16.387854  treatment     new\_page          0
         822059       UK  2017-01-16 14:04:14.719771  treatment     new\_page          1
         711597       UK  2017-01-22 03:14:24.763511    control     old\_page          0
\end{Verbatim}
            
    Criando as variáveis dummie:

    \begin{Verbatim}[commandchars=\\\{\}]
{\color{incolor}In [{\color{incolor}20}]:} \PY{n}{dfnovopaises2} \PY{o}{=} \PY{n}{dfnovopaises}\PY{o}{.}\PY{n}{join}\PY{p}{(}\PY{n}{pd}\PY{o}{.}\PY{n}{get\PYZus{}dummies}\PY{p}{(}\PY{n}{dfnovopaises}\PY{p}{[}\PY{l+s+s1}{\PYZsq{}}\PY{l+s+s1}{country}\PY{l+s+s1}{\PYZsq{}}\PY{p}{]}\PY{p}{)}\PY{p}{)}
         \PY{n}{dfnovopaises2}\PY{o}{.}\PY{n}{head}\PY{p}{(}\PY{l+m+mi}{4}\PY{p}{)}
\end{Verbatim}


\begin{Verbatim}[commandchars=\\\{\}]
{\color{outcolor}Out[{\color{outcolor}20}]:}         country                   timestamp      group landing\_page  \textbackslash{}
         user\_id                                                               
         834778       UK  2017-01-14 23:08:43.304998    control     old\_page   
         928468       US  2017-01-23 14:44:16.387854  treatment     new\_page   
         822059       UK  2017-01-16 14:04:14.719771  treatment     new\_page   
         711597       UK  2017-01-22 03:14:24.763511    control     old\_page   
         
                  converted  CA  UK  US  
         user\_id                         
         834778           0   0   1   0  
         928468           0   0   0   1  
         822059           1   0   1   0  
         711597           0   0   1   0  
\end{Verbatim}
            
    \begin{enumerate}
\def\labelenumi{\alph{enumi}.}
\setcounter{enumi}{7}
\tightlist
\item
  Though you have now looked at the individual factors of country and
  page on conversion, we would now like to look at an interaction
  between page and country to see if there significant effects on
  conversion. Create the necessary additional columns, and fit the new
  model.
\end{enumerate}

Provide the summary results, and your conclusions based on the results.

    Jogando fora uma das dummies para não dar \textbf{sobrevinculação}:

    \begin{Verbatim}[commandchars=\\\{\}]
{\color{incolor}In [{\color{incolor}21}]:} \PY{n}{dfnovopaises2} \PY{o}{=} \PY{n}{dfnovopaises2}\PY{o}{.}\PY{n}{drop}\PY{p}{(}\PY{l+s+s1}{\PYZsq{}}\PY{l+s+s1}{US}\PY{l+s+s1}{\PYZsq{}}\PY{p}{,} \PY{n}{axis} \PY{o}{=} \PY{l+m+mi}{1}\PY{p}{)}
         \PY{n}{dfnovopaises2}\PY{o}{.}\PY{n}{head}\PY{p}{(}\PY{l+m+mi}{4}\PY{p}{)}
\end{Verbatim}


\begin{Verbatim}[commandchars=\\\{\}]
{\color{outcolor}Out[{\color{outcolor}21}]:}         country                   timestamp      group landing\_page  \textbackslash{}
         user\_id                                                               
         834778       UK  2017-01-14 23:08:43.304998    control     old\_page   
         928468       US  2017-01-23 14:44:16.387854  treatment     new\_page   
         822059       UK  2017-01-16 14:04:14.719771  treatment     new\_page   
         711597       UK  2017-01-22 03:14:24.763511    control     old\_page   
         
                  converted  CA  UK  
         user\_id                     
         834778           0   0   1  
         928468           0   0   0  
         822059           1   0   1  
         711597           0   0   1  
\end{Verbatim}
            
    Criando o intercept:

    \begin{Verbatim}[commandchars=\\\{\}]
{\color{incolor}In [{\color{incolor}22}]:} \PY{n}{dfnovopaises2}\PY{p}{[}\PY{l+s+s1}{\PYZsq{}}\PY{l+s+s1}{intercept}\PY{l+s+s1}{\PYZsq{}}\PY{p}{]} \PY{o}{=} \PY{l+m+mi}{1}
         \PY{n}{dfnovopaises2}\PY{o}{.}\PY{n}{head}\PY{p}{(}\PY{l+m+mi}{4}\PY{p}{)}
\end{Verbatim}


\begin{Verbatim}[commandchars=\\\{\}]
{\color{outcolor}Out[{\color{outcolor}22}]:}         country                   timestamp      group landing\_page  \textbackslash{}
         user\_id                                                               
         834778       UK  2017-01-14 23:08:43.304998    control     old\_page   
         928468       US  2017-01-23 14:44:16.387854  treatment     new\_page   
         822059       UK  2017-01-16 14:04:14.719771  treatment     new\_page   
         711597       UK  2017-01-22 03:14:24.763511    control     old\_page   
         
                  converted  CA  UK  intercept  
         user\_id                                
         834778           0   0   1          1  
         928468           0   0   0          1  
         822059           1   0   1          1  
         711597           0   0   1          1  
\end{Verbatim}
            
    Não use: (só se precisar!)

    \begin{Verbatim}[commandchars=\\\{\}]
{\color{incolor}In [{\color{incolor}23}]:} \PY{n}{dfnovopaises2} \PY{o}{=} \PY{n}{dfnovopaises2}\PY{o}{.}\PY{n}{drop}\PY{p}{(}\PY{l+s+s1}{\PYZsq{}}\PY{l+s+s1}{intercept}\PY{l+s+s1}{\PYZsq{}}\PY{p}{,} \PY{n}{axis} \PY{o}{=} \PY{l+m+mi}{1}\PY{p}{)}
\end{Verbatim}


    \begin{Verbatim}[commandchars=\\\{\}]
{\color{incolor}In [{\color{incolor}24}]:} \PY{n}{dfnovopaises2}\PY{o}{.}\PY{n}{head}\PY{p}{(}\PY{l+m+mi}{4}\PY{p}{)}
\end{Verbatim}


\begin{Verbatim}[commandchars=\\\{\}]
{\color{outcolor}Out[{\color{outcolor}24}]:}         country                   timestamp      group landing\_page  \textbackslash{}
         user\_id                                                               
         834778       UK  2017-01-14 23:08:43.304998    control     old\_page   
         928468       US  2017-01-23 14:44:16.387854  treatment     new\_page   
         822059       UK  2017-01-16 14:04:14.719771  treatment     new\_page   
         711597       UK  2017-01-22 03:14:24.763511    control     old\_page   
         
                  converted  CA  UK  
         user\_id                     
         834778           0   0   1  
         928468           0   0   0  
         822059           1   0   1  
         711597           0   0   1  
\end{Verbatim}
            
    Claro que eu me esqueci do principal:

    \begin{Verbatim}[commandchars=\\\{\}]
{\color{incolor}In [{\color{incolor}30}]:} \PY{n}{dfnovopaises2}\PY{p}{[}\PY{p}{[}\PY{l+s+s1}{\PYZsq{}}\PY{l+s+s1}{control}\PY{l+s+s1}{\PYZsq{}}\PY{p}{,} \PY{l+s+s1}{\PYZsq{}}\PY{l+s+s1}{treatment}\PY{l+s+s1}{\PYZsq{}}\PY{p}{]}\PY{p}{]} \PY{o}{=} \PY{n}{pd}\PY{o}{.}\PY{n}{get\PYZus{}dummies}\PY{p}{(}\PY{n}{dfnovopaises2}\PY{p}{[}\PY{l+s+s1}{\PYZsq{}}\PY{l+s+s1}{group}\PY{l+s+s1}{\PYZsq{}}\PY{p}{]}\PY{p}{)}
         \PY{n}{dfnovopaises2} \PY{o}{=} \PY{n}{dfnovopaises2}\PY{o}{.}\PY{n}{drop}\PY{p}{(}\PY{l+s+s1}{\PYZsq{}}\PY{l+s+s1}{control}\PY{l+s+s1}{\PYZsq{}}\PY{p}{,} \PY{n}{axis} \PY{o}{=} \PY{l+m+mi}{1}\PY{p}{)}
\end{Verbatim}


    \begin{Verbatim}[commandchars=\\\{\}]
{\color{incolor}In [{\color{incolor}34}]:} \PY{n}{dfnovopaises2}\PY{o}{.}\PY{n}{head}\PY{p}{(}\PY{l+m+mi}{4}\PY{p}{)}
\end{Verbatim}


\begin{Verbatim}[commandchars=\\\{\}]
{\color{outcolor}Out[{\color{outcolor}34}]:}         country                   timestamp      group landing\_page  \textbackslash{}
         user\_id                                                               
         834778       UK  2017-01-14 23:08:43.304998    control     old\_page   
         928468       US  2017-01-23 14:44:16.387854  treatment     new\_page   
         822059       UK  2017-01-16 14:04:14.719771  treatment     new\_page   
         711597       UK  2017-01-22 03:14:24.763511    control     old\_page   
         
                  converted  CA  UK  treatment  
         user\_id                                
         834778           0   0   1          0  
         928468           0   0   0          1  
         822059           1   0   1          1  
         711597           0   0   1          0  
\end{Verbatim}
            
    \begin{Verbatim}[commandchars=\\\{\}]
{\color{incolor}In [{\color{incolor}37}]:} \PY{n}{dfnovopaises2}\PY{o}{.}\PY{n}{columns} \PY{o}{=} \PY{p}{[}\PY{l+s+s1}{\PYZsq{}}\PY{l+s+s1}{country}\PY{l+s+s1}{\PYZsq{}}\PY{p}{,} \PY{l+s+s1}{\PYZsq{}}\PY{l+s+s1}{timestamp}\PY{l+s+s1}{\PYZsq{}}\PY{p}{,} \PY{l+s+s1}{\PYZsq{}}\PY{l+s+s1}{group}\PY{l+s+s1}{\PYZsq{}}\PY{p}{,} \PY{l+s+s1}{\PYZsq{}}\PY{l+s+s1}{landing\PYZus{}page}\PY{l+s+s1}{\PYZsq{}}\PY{p}{,} \PY{l+s+s1}{\PYZsq{}}\PY{l+s+s1}{converted}\PY{l+s+s1}{\PYZsq{}}\PY{p}{,} \PY{l+s+s1}{\PYZsq{}}\PY{l+s+s1}{CA}\PY{l+s+s1}{\PYZsq{}}\PY{p}{,} \PY{l+s+s1}{\PYZsq{}}\PY{l+s+s1}{UK}\PY{l+s+s1}{\PYZsq{}}\PY{p}{,} \PY{l+s+s1}{\PYZsq{}}\PY{l+s+s1}{ab\PYZus{}page}\PY{l+s+s1}{\PYZsq{}}\PY{p}{]}
         \PY{n}{dfnovopaises2}\PY{o}{.}\PY{n}{head}\PY{p}{(}\PY{l+m+mi}{4}\PY{p}{)}
\end{Verbatim}


\begin{Verbatim}[commandchars=\\\{\}]
{\color{outcolor}Out[{\color{outcolor}37}]:}         country                   timestamp      group landing\_page  \textbackslash{}
         user\_id                                                               
         834778       UK  2017-01-14 23:08:43.304998    control     old\_page   
         928468       US  2017-01-23 14:44:16.387854  treatment     new\_page   
         822059       UK  2017-01-16 14:04:14.719771  treatment     new\_page   
         711597       UK  2017-01-22 03:14:24.763511    control     old\_page   
         
                  converted  CA  UK  ab\_page  
         user\_id                              
         834778           0   0   1        0  
         928468           0   0   0        1  
         822059           1   0   1        1  
         711597           0   0   1        0  
\end{Verbatim}
            
    Agora sim:

    \begin{Verbatim}[commandchars=\\\{\}]
{\color{incolor}In [{\color{incolor}38}]:} \PY{n}{dfnovopaises2}\PY{p}{[}\PY{l+s+s1}{\PYZsq{}}\PY{l+s+s1}{intercept}\PY{l+s+s1}{\PYZsq{}}\PY{p}{]} \PY{o}{=} \PY{l+m+mi}{1}
         \PY{n}{dfnovopaises2}\PY{o}{.}\PY{n}{head}\PY{p}{(}\PY{l+m+mi}{4}\PY{p}{)}
\end{Verbatim}


\begin{Verbatim}[commandchars=\\\{\}]
{\color{outcolor}Out[{\color{outcolor}38}]:}         country                   timestamp      group landing\_page  \textbackslash{}
         user\_id                                                               
         834778       UK  2017-01-14 23:08:43.304998    control     old\_page   
         928468       US  2017-01-23 14:44:16.387854  treatment     new\_page   
         822059       UK  2017-01-16 14:04:14.719771  treatment     new\_page   
         711597       UK  2017-01-22 03:14:24.763511    control     old\_page   
         
                  converted  CA  UK  ab\_page  intercept  
         user\_id                                         
         834778           0   0   1        0          1  
         928468           0   0   0        1          1  
         822059           1   0   1        1          1  
         711597           0   0   1        0          1  
\end{Verbatim}
            
    Instanciando:

    \begin{Verbatim}[commandchars=\\\{\}]
{\color{incolor}In [{\color{incolor}42}]:} \PY{n}{modlog2} \PY{o}{=} \PY{n}{sm}\PY{o}{.}\PY{n}{Logit}\PY{p}{(}\PY{n}{dfnovopaises2}\PY{p}{[}\PY{l+s+s1}{\PYZsq{}}\PY{l+s+s1}{converted}\PY{l+s+s1}{\PYZsq{}}\PY{p}{]}\PY{p}{,} \PY{n}{dfnovopaises2}\PY{p}{[}\PY{p}{[}\PY{l+s+s1}{\PYZsq{}}\PY{l+s+s1}{intercept}\PY{l+s+s1}{\PYZsq{}}\PY{p}{,} \PY{l+s+s1}{\PYZsq{}}\PY{l+s+s1}{ab\PYZus{}page}\PY{l+s+s1}{\PYZsq{}}\PY{p}{,} \PY{l+s+s1}{\PYZsq{}}\PY{l+s+s1}{CA}\PY{l+s+s1}{\PYZsq{}}\PY{p}{,} \PY{l+s+s1}{\PYZsq{}}\PY{l+s+s1}{UK}\PY{l+s+s1}{\PYZsq{}}\PY{p}{]}\PY{p}{]}\PY{p}{)}
\end{Verbatim}


    Ajustando:

    \begin{Verbatim}[commandchars=\\\{\}]
{\color{incolor}In [{\color{incolor}43}]:} \PY{n}{resultados2} \PY{o}{=} \PY{n}{modlog2}\PY{o}{.}\PY{n}{fit}\PY{p}{(}\PY{p}{)}
\end{Verbatim}


    \begin{Verbatim}[commandchars=\\\{\}]
Optimization terminated successfully.
         Current function value: 0.366113
         Iterations 6

    \end{Verbatim}

    \begin{Verbatim}[commandchars=\\\{\}]
{\color{incolor}In [{\color{incolor}44}]:} \PY{n}{resultados2}\PY{o}{.}\PY{n}{summary}\PY{p}{(}\PY{p}{)}
\end{Verbatim}


\begin{Verbatim}[commandchars=\\\{\}]
{\color{outcolor}Out[{\color{outcolor}44}]:} <class 'statsmodels.iolib.summary.Summary'>
         """
                                    Logit Regression Results                           
         ==============================================================================
         Dep. Variable:              converted   No. Observations:               290584
         Model:                          Logit   Df Residuals:                   290580
         Method:                           MLE   Df Model:                            3
         Date:                Thu, 07 Mar 2019   Pseudo R-squ.:               2.323e-05
         Time:                        23:10:31   Log-Likelihood:            -1.0639e+05
         converged:                       True   LL-Null:                   -1.0639e+05
                                                 LLR p-value:                    0.1760
         ==============================================================================
                          coef    std err          z      P>|z|      [0.025      0.975]
         ------------------------------------------------------------------------------
         intercept     -1.9893      0.009   -223.763      0.000      -2.007      -1.972
         ab\_page       -0.0149      0.011     -1.307      0.191      -0.037       0.007
         CA            -0.0408      0.027     -1.516      0.130      -0.093       0.012
         UK             0.0099      0.013      0.743      0.457      -0.016       0.036
         ==============================================================================
         """
\end{Verbatim}
            
    Adicionamos todas duas novas \textbf{dimensões} ao nosso modelo,
acrescentando a informação de \textbf{países}. Parece pouco? Pense que
se formos adicionando e adicionando novas dimensões, qual a complexidade
(e o ruído produzido) que atingiremos.

Se a dimensão fosse \textbf{relevante}, OK. Mas ao que parece, valor-p
acima do crítico de 5\% mostra que as novas feições não ajudaram a
explicar o fenômeno das conversões. Então neste caso, recomenda-se
\textbf{não manter} as novas feições, vindas da informação de
\textbf{países}.

Resumo da ópera: nem a mudança de página, da antiga para a nova, nem o
fato do usuário ter vindo de CA ou de UK parece explicar o fenômeno das
\textbf{conversões}.

    \hypertarget{conclusions}{%
\subsection{Conclusions}\label{conclusions}}

\begin{center}\rule{0.5\linewidth}{\linethickness}\end{center}

    \hypertarget{conclusuxe3o-geral}{%
\paragraph{Conclusão geral}\label{conclusuxe3o-geral}}

Com relação aos testes realizados, não houve evidência suficiente para
rejeitar a hipótese nula e as regressões feitas com relação à conversões
não parecem ser expicadas pela criação da nova página. Segundo os testes
aplicados, a nova página não deveria ser colocada em produção. Testado
por:

\begin{itemize}
\item
  distribuição amostral
\item
  regressão logarítmica
\end{itemize}

\hypertarget{possuxedveis-distoruxe7uxf5es}{%
\paragraph{Possíveis distorções}\label{possuxedveis-distoruxe7uxf5es}}

Os testes podem ter sofrido algumas \textbf{distorções} que poderiam
alterar seus resultados (efeito do tempo, efeito da novidade, etc..)

Ao invés de apenas citar potenciais \textbf{distorções}, citando as da
classe, resolvi criar uma lista bastante complexa de fenômenos que
poderiam gerar distorções. E algumas delas podem ser muito benéficas
para nossa nova estrutura. Isso extrapola o conteúdo dos testes A/B e se
baseia fortemente em estudos de \textbf{Teoria Geral de Sistemas} e em
\textbf{Cibernética}. Seguem alguns deles (dos fáceis aos mais complexos
para o final):

\begin{itemize}
\item
  \textbf{grupo de usuários que realizou os testes pode não ter sido o
  mesmo do que o grupo de usuários que irá fazer a compra}. Às vezes
  empresas colocam seus próprios funcionários, professores os seus
  alunos e programadores, os seus amigos (também programadores) para
  fazer testes em páginas Web. Bom, isso não é muito confiável, pois o
  domínio de crenças, as esferas de valores, a capacidade econômica e
  outros fatores explicados pela \textbf{Teoria dos Jogos} podem
  explicar distorções não hora de se se simular/comprar (eu posso
  arriscar 1 bilhão em ações num simulador e não ter coragem de comprar
  um título de R\$5.000,00 num banco e que envolva risco!)
\item
  \textbf{tempo do teste insuficiente} - Às vezes coisas levam um tempo
  para se estabilizar. Pessoas acostumadas com o antigo ``site'' se
  sentem constrangidas no novo (ou o contrário) e são tentadas a comprar
  mais, ou menos do que ao longo do tempo. As pessoas terão talvez que
  se acostumar com a nova plataforma
\item
  \textbf{engodo do ``estouro de vendas''} - (um exemplo disso é o
  \textbf{erro da promoção}: uma loja não está vendendo bem e resolve
  fazer uma ``queima de estoque''. Então aparecem muitos e muitos
  compradores. Tudo volta a se estabilizar e as vendas com o passar dos
  meses, se nota que estão \textbf{abaixo} das da queima\ldots{} por
  quê? Está todo mundo esperando o novo ``queimão''!)
\item
  \textbf{mercadorias tendenciosas} - Eu vendia muito bem no meu antigo
  ``site'' um ``Eau de Parfum'' que custa 500 o frasco de 50ml. No novo
  ``site'' eu vendo melhor uma ``Eau de Toilette'' que custa 125 o
  frasco de 250ml. Só que a minha margem de lucro nas antigas vendas era
  consideravelmente melhor! (Às vezes estamos em um \textbf{franja de
  mercado} e mesmo ao custo de vender pouco, evitamos uma concorrência
  desnecessária com \emph{O Boticário}, para dar um exemplo
\item
  \textbf{mudança de freguesia} - Pronto, eu peguei um público mais
  jovem e arrojado\ldots{} isso é mal? Não necessariamente! mas é o
  mesmo problema do ítem \textbf{anterior} (Se eu conseguir permissão
  para acessar o perfil do \emph{Facebook} do meu cliente, isso seria
  talvez o sonho!)
\item
  \textbf{custo da rejeição é maior do que o ganho da promoção} - Isso
  recentemente virou uma discussão envolvendo entre outras coisas
  \textbf{eleição presidencial} (não entrando no mérito, pois aqui isso
  é \textbf{irrelevante}!)\ldots{} se a antiga estrutura me causava
  pouca rejeição, mas poucas vendas\ldots{} a estrutura nova me causa
  muitas vendas mas maior rejeição\ldots{} isso é uma balança
  \textbf{delicadíssima} e que pode me custar o meu comércio fechar as
  portas em um futuro próximo\ldots{} altas taxas de rejeição, seja lá
  no que for, tendem a dar \textbf{Efeito de Memória} e criar maiores e
  maiores rejeições a longo prazo\ldots{}
\item
  por falar em efeito de memória\ldots{} vale a pena citar
  \textbf{Fenômeno de Aniversariamento} - Aniversariamento é um fator da
  memória humana muito estudado por psiquiatras e analistas do
  comportamento humano. Basicamente funciona assim: num país hipotético,
  um presidente hipotético mudou o padrão monetário para \(Cruzado\) a
  fim de evitar inflação galopante. Junto com este pacote, que a
  princípio foi \textbf{bem recebido} pela população, o gentil homem
  teria feito um confisco de poupança hipotético, que gerou grande
  instatisfação e consequentemente, uma movimentação popular que o levou
  a perder o cargo de presidência hipotética. O que aconteceria se um
  novo presidente, para conter evolução inflacionária propusesse mudança
  no padrão monetário? Quase que certamente uma \textbf{revolta popula
  imediata e sem precedentes}. Qual a razão? \textbf{Fenômeno de
  Aniversariamento}. E isso poderia ocorrer, caso tentássemos diversos
  ``sites'' novos e problemáticos para nossa \textbf{Audacity}\ldots{}
\item
  \textbf{Efeito Zeitgeist} (``Espírito do Tempo'') - Era moda no
  entre-guerras a moda ``Prêt-a-porter'' (pronta para sair), um estilo
  provocante e inovador da mulher jovem dizendo: ``meus pais não me
  seguram mais, acabou a guerra e com ela, a tradição\ldots{}''. A
  escola Bauhaus foi considera \textbf{socializante e escandalosa} e em
  1933, foi fechada à força por uma sociedade que buscava valores
  mais\ldots{} ``tradicionais'' na Alemanha Nazista\ldots{} E hoje a
  maioria do nosso mobiliário possui \textbf{forte inspiração} na
  Bauhaus! Então nosso ``site'' pode receber mais conversões agora por
  ``ser vintage'' e alguns anos depois ser considerado ``boco-moco''
  pela maioria dos seus usuários\ldots{} mesmo sem \textbf{nenhuma}
  modificação!
\item
  \textbf{Efeito Repertório} - É um outro fenômeno estudado por
  psiquiatras e analistas de comportamento. Pessoas com uma cultura mais
  limitada (independendo se vieram de castas economicamente mais
  favorecidas, ou de regiões consideradas mais ricas de um país), quando
  são expostas a \textbf{situações novas} tendem a ficar sem ação, ou
  mesmo entrar em desespero. Suponha um exemplo de dois engenheiros.
  Ambos fizeram faculdade de Engenharia, se formaram e foram trabalhar.
  Mais tarde houve uma crise fenomenal e ambos ficaram sem
  emprego\ldots{} o detalhe é que um deles aprendeu o ofício de
  \textbf{marcenaria} na sua juventude e com isso conseguiu pagar as
  contas fazendo banquinhos para vender na feira\ldots{} um
  \textbf{indígena} ou um \textbf{quilombola} pode ter mais repertório
  do que um cidadão moderno médio e isso pode lhe dar mais chances de
  adaptação/sobrevivência em condições extremas\ldots{} bom, se eu
  trabalho com um público de \textbf{pouco repertório}, qualquer mudança
  tenderá a ser mal vista e criticada, mesmo que seja para o bem\ldots{}
  como saber se meu público é assim? Isso demanda análises mais apuradas
  e complexas\ldots{}
\item
  \textbf{Efeito Aprendizado} - Ao longo do tempo, eu aprendo a corrigir
  e a limar meu ``site'' e a manutenção passa a custar pouco e ser mais
  fácil\ldots{} já o meu novo ``site'' funciona mais como um
  \textbf{protótipo} de um avião ou de um automóvel\ldots{} ele tende a
  me dar mais problemas, a estar com mais arestas\ldots{} tudo isso
  desagrada o usuário\ldots{} mesmo que ao final meu produto seja muito
  melhor!
\item
  \textbf{Efeito Vaca Sagrada} - Na Índia, as vacas eram consideradas
  animais sagrados. Mas um proprietário de uma empresa, ou antigos
  diretores com muita voz na decisão podem dizer ``Ah\ldots{} isso não é
  negociável! Isso é \textbf{sagrado}!''. Então pronto, nós temos
  algumas feições do nosso projeto que são \textbf{inalteráveis} (e que
  podem tornar muito cara minha mudança para o novo ``site'')
\item
  \textbf{Efeito Ação Imediata} - Se percebeu que quando colocado sob
  pressão, o ser humano costuma a optar por ferramentas intelectuais
  mais primitivas e que dão respostas mais imediatas. Assim, se eu tenho
  que montar um caixote de madeira e tenho meu final de semana para
  isso, primeiro eu abro minha latinha de cerveja, separo as ferramentas
  (chave de fenda, furadeira, parafusos para madeira, cola\ldots{}),
  faço as medições, desenho no papel\ldots{} agora se é para um ``se
  vira nos 30!'', eu nem consigo pensar direito, vai ser no serrote,
  prego e martelo! Então, se eu induzir meu cliente a executar sua
  compra na tela ``promoção relâmpago: 30 segundos'', eu possivelmente
  acionarei partes mais primitivas do cérebro dele e ele provavelmente
  fará a compra\ldots{} e 15 minutos depois estará me ligando para
  cancelar por arrependimento! Má estratégia! Se meu ``site'' novo se
  basear nesta ação, estarei perdido\ldots{}
\item
  \textbf{Efeito em Cadeia} - Um curso leva a outro e isso leva a
  outro\ldots{} e a um programa de fidelização\ldots{} Se meu ``site''
  novo refletir melhor isso, eu poderei esperar lucros maiores e
  seguros\ldots{} num horizonte temporal maior. É difícil de modelizar
  isso\ldots{} mas suponha que minha \textbf{Audacity} agora oferece um
  determinado curso hipotético chamado \textbf{Fundamentos em Data
  Science II} a ser feito como continuação do\ldots{}
\item
  \textbf{Perfis Estratégicos} - Em \textbf{Teoria dos Jogos} temos
  jogadores \textbf{Aversos ao Risco}, \textbf{Neutros ao Risco}, etc..
  No comércio, vendedores costumam classificar pessoas como
  \textbf{Apoiador}, \textbf{Meticuloso}, \textbf{Realizador} e
  \textbf{Expressivo}. Às vezes o novo ``site'' agrada a um desses
  grupos, mas causa problemas com os outros!
\item
  \textbf{Jogo do Prisioneiro} - Um clássico problema de \textbf{Teoria
  dos Jogos}, uma determinada pessoa, sem saber qual foi a decisão das
  demais, é levada a ``converter'' ou ``não converter'', segundo um
  cálculo de riscos e benefícios potenciais. A \textbf{Audacity} poderia
  usar uma estratégia deste tipo na sua nova página\ldots{}
\item
  \textbf{Tac-for-Toe} ou ``jogo da manipulação''- Analisando
  comportamento de pássaros, se descobriu que é muito frequente que um
  deles, para induzir determinado comportamento nos demais, faça uma
  determinada ação. Então meu novo ``site'' me ``oferece uma minhoca'',
  na esperança de que o cliente faça o mesmo no seu turno. É um
  comportamento complexo e que em alguns casos, dá muito certo
  (golpistas e fraudadores costumam usar com frequência desta tática,
  mas ela poderia ser usada para o bem).
\item
  \textbf{Efeito Titanic} - Eu já investi 100 mil na minha antiga página
  da \textbf{Audacity}\ldots{} como eu justificarei isso para meus
  diretores? Não seria melhor gastar os 20 mil para deixá-la ainda
  melhor? Nem é necessário explicar muito, mas o \textbf{Efeito Titanic}
  costuma causar muitas distorções na hora de avaliar resultados
  \textbf{reais}, como a do desempenho da minha nova plataforma\ldots{}
\item
  \textbf{Efeito de Clã} - Eu uso um jargã próprio (como num grupo
  hipotético de Twitter envolvendo \textbf{Floofer},
  \textbf{Doggo}\ldots{}) e com isso eu atraio apenas pessoas do meu
  próprio clã. Isso pode ser bom ou muito \textbf{prejudicial} (o que
  normalmente é) nas feições do meu novo ``site''\ldots{}
\item
  \textbf{Efeito Iniciático} - O meu aluno do meu novo ``site'' é
  compelido a se sentir um ``iniciado'', com provas de iniciação e tudo
  mais na minha nova estrutura\ldots{} e ele é compelido a comprar novos
  produtos, cada vez mais complexos e caros\ldots{} (isso parece
  extremamente \textbf{maléfico}, mas nem sempre! - às vezes a sociedade
  acaba reconhecendo o aluno da \textbf{Audacity} como um cara valoroso
  para se ter numa empresa e os ganhos individuais podem superar e muito
  o valor investodo\ldots{})
\item
  \textbf{Efeito Moby Dick} - Inspirado no romance de Herman Melville,
  um indivíduo (o capitão Ahrab) se torna, após um trauma (perda de uma
  perna), tão obstinado a eliminar um determinado cachalote (Moby Dick)
  que acaba levando seu navio e tripulaão ao desastre. Em alguns casos,
  o meu ``site'' pode resultar em ações muito diversas do esperado, ao
  expor seus integrantes a situações traumáticas (como um hipotético
  curso imenso de \textbf{Data Science 2} e que parece não ter mais
  fim!)
\item
  \textbf{Efeito Dominó} - Minha nova página altamente promocional,
  oferecendo aquele curso dos sonhos por ``apenas R\$5,00'' fez com que
  a \emph{Udemy}, a \emph{Udacity}, meus concorrentes, a fazerem o
  mesmo. Resultado: um leilão de cursos baixos que fará todo mundo
  perder dinheiro (e os mais fracos a falirem). Os pontos de equilíbrio
  de \textbf{Nash}, em \textbf{Teoria dos Jogos} ajudam a descrever este
  e outros efeitos de grupo e que podem ter resultados terríveis.
\item
  \textbf{Hipnose} - Não se sabe exatamente como ela funciona.
  Aparentemente algumas pessoas já são \textbf{sugestionáveis} para
  tomar determinado tipo de ação. Foram feitos testes exaustivos neste
  tema nos anos 60 e 70 e nunca se chegou a uma conclusão clara (pessoas
  fortemente hipnotizadas foram apresentadas a um copo de água comum,
  que lhes afirmaram conter um ácido terrível. Elas agiram de maneira
  compatível à sugestão e quando pedidas a \textbf{arremessar o ácido}
  ao rosto de outra pessoa, elas o faziam com facilidade. No entanto,
  quando aparecia uma pessoa com luvas e máscara e enchia com cautela um
  segundo copo de solução salina parecida com ácido de verdade, o que
  ocorria era invariavelmente a pessoa sair do \textbf{transe
  hipnótico}). Explorar estes efeitos num site da \textbf{Audacity}?
  Isso poderia custar \textbf{boas risadas} ou um enorme
  \textbf{processo judicial}. É anti-ético manipular pessoas! (embora
  diversas seitas façam isso a todo momento!).
\item
  \textbf{Teoria da Guerra} - Um dos campos pouco estudados na Teoria
  dos Jogos. Mas o importante ao final é \textbf{ganhar todas as
  batalhas}, \textbf{subjugar e humilhar o oponente}, \textbf{não sofrer
  muitos danos} ou \textbf{estabelecer a paz a qualquer preço}? Ao criar
  uma nova plataforma em um mercado competitivo, a nossa
  \textbf{Audacity} terá que lidar com todos esses dilemas. E isso não é
  explicado por um simples \textbf{Teste de Hipóteses}\ldots{}
\end{itemize}

\emph{(Tirado a grosso modo dos meus estudos em Sistêmica, na leitura da
``Enciclopédia Internacional de Cibernética e Sistêmica'', de Charles
François, 2 vols.)}

Tudo isso deveria ser levado em consideração ao se fazer projetos para
lançamento de novos produtos\ldots{}


    % Add a bibliography block to the postdoc
    
    
    
    \end{document}
